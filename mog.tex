%!TEX TS-program = xelatex 
%!TEX TS-options = -synctex=1 -output-driver="xdvipdfmx -q -E"
%!TEX encoding = UTF-8 Unicode
%
%  mog
%
%  Created by Mark Eli Kalderon on 2010-04-06.
%  Copyright (c) 2010. All rights reserved.
%

\documentclass[12pt]{article} 

% Definitions
\newcommand\mykeywords{given, Sellars, perception, reasons} 
\newcommand\myauthor{Mark Eli Kalderon} 
\newcommand\mytitle{Before the Law}

\input{preamble/preamble.tex}

%%% BEGIN DOCUMENT
\begin{document}

% Title Page
\maketitle
% \begin{abstract} % optional
% \noindent
% \end{abstract} 
\vskip 2em \hrule height 0.4pt \vskip 2em
% Main Content
\epigraph{Before the law sits a gatekeeper. To this gatekeeper comes a man from the country who asks to gain entry into the law. But the gatekeeper says that he cannot grant him entry at the moment. The man thinks about it and then asks if he will be allowed to come in sometime later on. “It is possible,” says the gatekeeper, “but not now.”---Franz Kafka}

% Layout Settings
\setlength{\parindent}{1em}

\section{Introduction} % (fold)
\label{sec:introduction}
\noindent It is notoriously difficult to understand the charge of falling prey to the Myth of the Given. The principal obstacle was erected by Sellars himself. The Myth of the Given is said to arise in various forms, and yet each of these forms participates in a framework of givenness---a framework that Sellars declines to specify in general terms. As interpreters of Sellars, we may speculate about how best to understand the general framework of givenness. It is reasonable to attempt to reconstruct that framework in light of Sellars' discussion of the more specific forms of the Myth and especially in light of the conceptual genealogies he offers. However, there is no consensus on how the Myth of the Given is to be understood in general terms even among philosophers sympathetic with Sellars on fundamental issues. Thus different accounts of the Myth are given by \citet{Bonjour:1985sj}, \citet{Brandom:2002fk}, \citet{Coates:2007ly}, \citet{deVries:2005qf}, \citet{McDowell:1996uq,McDowell:1998vn}, \citet{Rosenberg:2007ve}, \citet{Rorty:1979ys}, and \citet{Williams:1977zr}.

Nevertheless, there may be occasions where one can know, or at least reasonably judge, that a conception of perception is a form of the Myth of the Given (if it is indeed a myth) without knowing in general terms what the Myth of the Given is. So, for example, I am increasingly attracted to a conception of visual perception where vision affords the perceiver with a sensory mode of awareness of particulars spatially distant from the perceiver's body. This distinctive sensory mode of awareness is non-propositional---\-it doesn't take a fact as its object, but a particular. Moreover this sensory mode of awareness is fundamental and irreducible. This fundamental and irreducible sensory mode of awareness of spatiotemporal particulars arrayed in a mind-independent environment makes one knowledgeable of that environment. In being so aware of a particular, the perceiver is in a position to know certain things about it, depending, of course, on the perceiver's possession and exercise of recognitional capacities appropriate to the given circumstances. We can be confident that Sellars, at least, would regard this as a form of the Myth of the Given since it involves a pre-conceptual mode of awareness.

Suppose, then, one is liable to the charge of falling prey to the Myth of the Given. How might one respond to the charge? Not directly, it seems. For suppose one could argue that the Myth of the Given was no myth on a specific understanding of that charge. Given the lack of consensus on how the framework of givenness is to be understood, it would remain an open question whether the conception defended was yet another form of the Myth on a distinct and potentially superior understanding of it. A lack of a definitive statement of the Myth from Sellars and a lack of consensus on how best to reconstruct the general framework of givenness renders the charge elusive and Protean. Being liable to the charge is like being the man from the country who seeks access to the law in Kafka's parable. Denied access to law, should one simply despair before its gatekeepers?

While no direct response is possible, perhaps an \emph{indirect} response is. I can think of two complementary strategies. First, one might strive to satisfy oneself that perception, so conceived, is epistemically significant---that the conception renders intelligible that perception should make one knowledgeable of a mind-independent subject matter. Of course, by itself, this would fail to persuade anyone inclined to prosecute that charge, but persuading the prosecution is unnecessary to achieve the modest task of setting one's own mind at ease about the epistemic significance of perception as one conceives it to be. The task may be modest in this way, but it is an essential first step that promises to clarify what's at issue in the dispute. Second, one might examine rival conceptions of perception that are tailor-made to avoid the Myth of the Given such as those provided by \citet{Sellars:1956xp} and \citet{McDowell:1996uq,McDowell:1998vn,McDowell:2008fk}. If criticism of conceptions designed specifically to avoid the Myth naturally motivates a conception that is known, or reasonably judged, to be a form of the Myth, this is some reason, at least, to think that Myth of the Given is no myth. 

These, then, are the strategies that I will be pursuing. In the first part, I proceed dogmatically, sketching how visual perception, conceived as a non-propositional sensory mode of awareness, can make the subject knowledgeable of particulars without the mind. In the second part, I proceed dialectically, discussing Sellars and McDowell on perceptual experience. Of particular interest will be McDowell's criticism that Sellars misconceives the nature of sensibility by allowing receptivity to operate independently of the subject's conceptual capacities. There are distinguishable claims made in the course of leveling this criticism. Marking these allows us to see that McDowell is right in rejecting Sellars' account, but wrong in thinking that this requires our conceptual capacities to be operative in perceptual awareness.

% section introduction (end)

\section{Sensory Awareness} % (fold)
\label{sec:sensory_awareness}

Dissatisfaction with the quality of tomatoes available in central London and not a concern for the veracity of philosophical clichès was my reason for growing tomatoes by my window. It is a large plant in a terra cotta pot that sits on a wooden stool set against a generous Victorian window. In the background is a field of rubble where Middlesex Hospital used to be. Everything has been leveled except for the chapel. Looking up I see a ripening tomato set against this complex scene. It is not quite ripe---it is a yellowish red, if not orange. Moreover, it is dappled in sunlight that has just emerged from the clouds in the aftermath of a shower. When I see the ripening tomato, the ripening tomato is the object of my perceptual experience---I am aware of the tomato in my perceptual experience of it. 

The awareness involved in my visual perception is \emph{sensory} in the way that the awareness of a passing thought in the stream of consciousness is not. While perceptual awareness is a sensory, sensory awareness need not be perceptual. Pain and proprioception are sensory but are plausibly nonperceptual. Moreover, vision affords the subject with a distinctive \emph{mode} of sensory awareness. Vision is not alone among the senses in providing information about the distal environment. Thus we can \emph{see} the leaves rustling and \emph{hear} them rustling. Perhaps, as \citet[]{Berkeley:1734fk} urged, we hear, at least in the first instance, the sound of the rustling leaves. But, if we do, then, at least in propitious circumstances, we hear the source of the sound by hearing the sound. So whereas vision affords the subject with a \emph{visual} mode of awareness of the event, arguably at least, audition affords the subject with an \emph{auditory} mode of awareness of the very same event. (For recent discussion of audition see Nudds and O'Callaghan \citeyear{Nudds:2009sk}).

The objects of visual awareness are spatiotemporal particulars arrayed in a mind-independent environment. These particulars do not constitute a unified ontological category. Among them are objects (the plant, the stool), events (the rustling of the leaves, the sun's setting), and property instances (the yellowish red of the tomato). But objects, events, and property instances have distinct modes of being. Thus, for example, events are spatiotemporal particulars that unfold through time in a way that neither objects nor property instances do (though see Sider \citeyear{Sider:1997fk} for criticism of this way of distinguishing events from objects; see also Hawthorne \citeyear{Hawthorne:2008uq} and Fine \citeyear{Fine:2006fk} for further discussion). The particulars that we see also differ in degree of both substantiality and being. In addition to ordinary material substances like tomatoes, we see flashes, flames, reflections, and rainbows, on the one hand, and shadows and holes, on the other hand. Flashes are unusual events. They are colored, but most events are not colored despited having colored participants. (What color was the Battle of Kosovo?) But what is presently important is that they also lack substrata. And the plant's shadow is constituted by a relative decrease in the amount of light uniformly distributed across the visible spectrum in a region determined by the positions of the plant, the light source, and the surface on which it is cast. But a relative decrease is a privation or diminution of being \citep[see][]{Sorensen:2008kx}. Shadows may be creatures of darkness, but that does not make them candidates for elimination. \citet{Quine:1948ef} may crack wise about the possible fat man in the door way, but perception makes shadows, a species of privation, the subject matter of demonstrative thought and talk. Despite being particulars, the objects of perception differ, in this way, in mode, substantiality, and being. (On the heterogeneity of the objects of visual perception see Austin \citeyear{Austin:1962lr}.)

Vision makes spatially distant particulars visually present. When I see the yellowish red tomato, I see a particular spatially distant from me. Moreover, I experience the yellowish red as inhering in the external surface of that particular. Hering's \citeyearpar[8]{Hering:1920ty} ringed shadow experiment demonstrates the phenomenological difference between a color merely overlaying a surface and that color inhering in that surface. When a shadow is cast on a white surface you see the shadow as ``an incidental darkness that lies on the [surface]''. If now a black line is drawn around the shadow completely obscuring its penumbra, the darkness no longer appears to lie on the surface, but rather appears to inhere in the surface. Only then does the shadowed white surface misleadingly look to be gray. In seeing the yellowish red tomato, I experience the yellowish red as inhering in the external surface of the tomato. Of course, not all colors inhere in substances or their parts. A flash can be blue even though the flash lacks a substrata in which the blue could inhere. But even if the blue of the flash inheres in nothing, I experience the blue in the remote spatiotemporal region of the flash. 

Thus, Broad writes:
\begin{quote}
    In its purely phenomenological aspect \emph{seeing} is ostensibly \emph{saltatory}. It seems to leap the spatial gap between the percipient's body and a remote region of space. Then again, it is ostensibly \emph{prehensive} of the surfaces of distant bodies as coloured and extended, and of external events as colour-occurences \emph{localized} in remote regions of space. \citep[32]{Broad:1965dq}
\end{quote}
Seeing is saltatory in that the objects of visual awareness are spatially remote; seeing is prehensive in that its objects are present in our visual awareness of them. ``Prehensive'' belongs to a primordial family of broadly tactile metaphors for visual awareness that includes ``grasping'', and ``apprehending''. What unites these metaphors is that they are all a mode of taking in, and ``ingestion'' is a natural variant \citep[see][7]{Johnston:2006uq,Price:1932fk}. The tactile nature of these metaphors can mislead, however, if we take too seriously the contact involved in taking in an object. Thus Broad remarks that ``It is a natural, if paradoxical, way of speaking to say that seeing seems to `bring us into \emph{contact} with \emph{remote} objects' and to reveal their shapes and colors''. The air of paradox, however, is removed once we recognize that the sense in which visual awareness brings us into contact with particulars does not conflict with the sense in which these particulars are remote. To be sure, seeing the tomato a meter away does not make the tomato proximate, but that just means that the sense in which the tomato is present in visual awareness is not the same sense as the tomato being spatially present.

The phenomenology of being visually presented with colors inhering in external bodies played a role in Russell's felt exuberance in abandoning idealism:
\begin{quote}
	I felt it, in fact, as a great liberation, as if I had escaped from a hothouse on to a wind-swept headland. I hated the stuffiness involved in supposing that space and time were only in my mind. I liked the starry heavens even better than the moral law, and could not bear Kant’s view that the one I liked best was only a subjective figment. In the first exuberance of liberation, I became a naïve realist and rejoiced in the thought that grass is really green, in spite of the adverse opinions of all philosophers from Locke onwards. \citep[48]{Russell:1959fv}
\end{quote}
What Russell rejoiced in was warrantedly taking external bodies to be the way his perceptual experience presents them to be---as having colors inhering in the opaque surfaces they present. He rejoiced in the greenness of the grass revealed by sight. Compare Melville's earlier revulsion at Locke's metaphysics:
\begin{quote}
	And when we consider that other theory of the natural philosophers, that all other earthly hues---every stately or lovely emblazoning---the sweet tinges of sunset skies and woods; yea, and the gilded velvets of butterflies, and the butterfly cheeks of young girls; all these are but subtile deceits, not actually inherent in substances, but only laid on from without; so that all deified Nature absolutely paints like the harlot, whose allurements cover nothing but the charnel-house within \ldots\ \citep[ch. 42]{Melville:1851ms}
\end{quote}
Freed from the adverse opinions of all philosophers from Locke onwards, one need no longer think that Nature paints like a harlot and can once again rejoice in the lovely emblazoning of an English pastoral scene. The phenomenology of being presented with colors inhering in external bodies renders intelligible this pattern of response.

Visual perception involves a sensory mode of awareness that takes external particulars as objects. This is a claim about sense perception, not sense experience (understood as the genus of which sense perception is a species). It echoes a conception of perception common among early twentieth century realists. There are, of course, important differences between the present conception and its early twentieth century precursors, just as there were important differences between the Cambridge and Oxford realists. Thus, contra the present conception, \citet[]{Russell:1912uq} held that we are acquainted with universals as well as particulars. And \citet[]{Russell:1912uq} and \citet[]{Price:1932fk} held that the distinctive sensory mode of awareness was operative not just in sense perception but in sense experience more generally. They were thus committed to a kind of \emph{experiential monism} (in Snowdon's \citeyear{Snowdon:2008oz} terminology). Specifically, all sense experience was conceived to involve, as part of its nature, a sensory mode of awareness. According to the Cambridge realists, then, even subject to illusion or hallucination, there is something of which one is aware. But with that, they were an application of the argument from illusion, or hallucination, or conflicting appearances away from immaterial sense data and a representative realism that tended, over time, to devolve into a form of phenomenalism. In this regard, I follow the salutary example of the Oxford realists, \citet{Cook-Wilson:1926sf} and \citet[]{Prichard:1906gf,Prichard:1909yg}, in restricting sensory awareness to perception, thus avoiding experiential monism and its degenerative effects \citep[see][]{Marion:2000kl,Marion:2000ai,Kalderon:2010fk}.

Sellars observes that if sensory awareness takes particulars as objects, then it cannot be a form of propositional knowledge:
\begin{quote}
	[W]e may well experience a feeling of surprise on noting that according to sense-datum theorists, it is \emph{particulars} that are sensed. For what is \emph{known}, even in in non-inferential knowledge, is \emph{facts} rather than particulars, items of the form \emph{something's being thus and so} or \emph{something's standing in a certain relation to something else}. It would seem, then, that the sensing of sense contents \emph{cannot} constitute knowledge, inferential \emph{or} non-inferential \ldots\ \citep[§3]{Sellars:1956xp}
\end{quote}
What's known are thoughts, propositions, facts---not particulars. Sellars doesn't give us a reason for distinguishing thoughts and particulars in this way. Perhaps he regards the distinction as evident. Prichard, however, gives us a reason, indeed, the right kind of reason. According to Prichard, thoughts have a kind of generality that precludes them from being particulars:
\begin{quote}
	There seems to be no way of distinguishing perception and conception as the apprehension of different realities except as the apprehension of the individual and the universal respectively. Distinguished in this way, the faculty of perception is that in virtue of which we apprehend the individual, and the faculty of conception is that power of reflection in virtue of which a universal is made the explicit object of thought. \citep[]{Prichard:1909yg}
\end{quote}
Prichard's claim about the faculty of conception is what provides a reason for distinguishing thoughts and particulars, and Sellars could accept that reason even if he would reject Prichard's claim about the faculty of perception. (For contemporary discussion of particularity and the content of perception see Brewer \citeyear{Brewer:2008fk}, Martin, \citeyear{Martin:2002jb} Soteriou \citeyear{Soteriou:2000iz,Soteriou:2005fk}, and Travis \citeyear{Travis:2005ys}.)

But suppose we accept Prichard's claim that ``the faculty of perception is that in virtue of which we apprehend the individual''. If visual awareness takes particulars as objects, it is not a form of propositional knowledge. But that does not mean that vision, so conceived, lacks epistemic significance. Vision can be a source of knowledge insofar as the perceiver can recognize the object of perception for what it is. When I look at the ripening tomato, the tomato is present in my awareness of it. Moreover, if I possess the appropriate recognitional capacities, in being so aware of the ripening tomato, I can come to know various things about it---that it is yellowish red, say. The non-propositional sensory mode of awareness involved in visual perception may not be a form of knowledge, but it makes me \emph{knowledgeable} of its object. Sensory awareness makes the subject knowledgeable of its object in the sense that it makes knowledge about its object \emph{available} to the perceiving subject. Vision would make the subject knowledgeable of its object even if, in the circumstances of perception, the subject lacked the conceptual capacities for knowing some range of propositions. Suppose the subject subsequently acquired the relevant conceptual capacities, in recalling the object of sight, the subject might recognize what they had seen and thus acquire propositional knowledge. So even if in the circumstances of perception the subject lacks the relevant conceptual capacities for knowing some proposition, vision can make knowledge of that proposition available. Perception constitutes a change in the subject's knowledge potential whether or not such knowledge is in fact activated (in Williamson's \citeyear{Williamson:1990uq} terminology).

In looking at the ripening tomato by my window, I see the yellowish red of the tomato. If I recognize what I thus see, I can come to know that the tomato is yellowish red. What I know is a proposition---that the tomato is yellowish red. What I see is a particular---the yellowish red of the tomato. How can awareness of that particular make available knowledge of that proposition? The objects of perceptual awareness are epistemically significant because of an alethic connection between them and potentially known propositions. The yellowish red of the tomato is a \emph{truthmaker} of the proposition that the tomato is yellowish red \citep[see][]{Johnston:2006uq}. It is impossible for the yellowish red of the tomato to exist and the proposition that the tomato is yellowish red to be false. More generally, I accept a version of \emph{truthmaker necessitarianism}:
\begin{quote}
	If a particular, \( x \)---be it an object, event, or property instance---is a truthmaker for a proposition, \( p \), then it is necessary that if \( x \) exists, then \( p \) is true. 
\end{quote}
\citep[for a survey of recent work on truthmakers see][]{Rodriguez-Pereyra:2006fk} What is the nature of the modality involved in this claim? I am tempted to say that it is logical necessity, though, of course, a sense of logical necessity distinct from model-theoretic definitions associated with formal systems  \citep[for doubts about model-theoretic definitions of logical necessity see][]{Etchemendy:1988et,Etchemendy:1988sa}. Nevertheless, the relevant modality deserves to be described as logical necessity, at least in a broader sense, since the source of the necessity flows from the nature of truth. (It is partly for this reason that I doubt that this could be the basis of a modal analysis of truthmaking.) Visual awareness is epistemically significant insofar as its object is the truthmaker of potentially known propositions. Vision makes me knowledgeable of particulars arrayed in a mind-independent environment by making me aware of sensible truthmakers.

A residual doubt may remain. Sellars famously connects knowledge and reasons in the following fashion:
\begin{quote}
	The essential point is that in characterizing an episode or state as that of \emph{knowing}, we are not giving an empirical description of that episode or state; we are placing it in the logical space of reasons, of justifying and being able to justify what one says. \citep[§36]{Sellars:1956xp}
\end{quote}
Sellars undertakes a number of separable commitments in this passage, but suppose at least this much---that we can only know what we have reason to judge. It is natural to think that vision is a potential source of knowledge only insofar as it makes the subject aware of what reasons there are. But now suppose further that reasons must have a propositional structure---either by being, or being individuated by, true propositions, or by being states or events with true propositional contents \citep[see][141, 143--4]{McDowell:1996uq}. Insofar as visual awareness takes particulars as its objects it lacks a propositional structure. But this would preclude it from the space of reasons. Seeing the tomato, at least as presently conceived, could not be an awareness of what reasons there are and so could not be a potential source of knowledge.

Not all philosophers accept that knowledge requires the possession of reasons \citep[see][]{Pryor:2007fk}. But even granting that knowledge involves reasons, the argument can be resisted insofar as it involves a substantive and controversial claim about the metaphysics of reasons---that reasons must have a propositional structure. Against this, let me dogmatically assert my adherence to a radically externalist conception of reasons \citep[see][]{Scanlon:1998hb,Raz:2000tm}. The conception is \emph{externalist} in that, according to it, reasons need not be propositional attitudes or any other psychological state of a subject. External reasons may not be propositional attitudes, but they may still have a propositional structure. Suppose reasons are facts. Facts are not psychological states of a subject, but they have the structure of the true propositions that represent them. What makes the conception \emph{radically} externalist is that, according to it, non-psychological reasons need not have a propositional structure.

On this conception, the yellowish red of the tomato is a reason for thinking that the tomato is not quite ripe. Note well, it is the yellowish red of the tomato, and not my seeing that the tomato is yellowish red, or my believing that the tomato is yellowish red, which is a reason. The yellowish red of the tomato lacks a propositional structure---it is a particular, a property instance. It is spatially distant from me---the yellowish red inheres in the opaque surface of the tomato a meter away and inherits its location from the surface in which it inheres. It is an aspect of how things are independently of me. The yellowish red of the tomato is a reason that warrants judging that the tomato is yellowish red. Indeed, in this instance, there could be no better reason---the yellowish red of the tomato warrants judging that the tomato is red because the former makes true the latter. Of course, not all reasons are of this form. The yellowish red of the tomato is also a reason, in certain circumstances, for thinking that the tomato is unripe, but the yellowish red of the tomato doesn't make it true that the tomato is unripe; at best, it is sign or symptom of the fruit's relative maturity. From this perspective, Davidson's \citeyearpar[310]{Davidson:1986uq} claim that ``nothing can count as a reason for holding a belief except another belief'' seems like an unwarranted form of psychologism (in something like Travis' \citeyear{Travis:2006fk} sense). Of course, in order for the yellowish red of the tomato to rationally bear on what I am to think, it must be cognitively accessible. But that is what perception does for me---perception makes me aware of what reasons there are. Perception is thus a mode of reasonableness (in Scanlon's \citeyear[]{Scanlon:1998hb} sense; for further relevant discussion see Ayers \citeyear{Ayers:2004kx} and McDowell \citeyear{McDowell:2006vn}).

Vision affords awareness of spatially distant particulars. In being so aware of a particular, the subject is in a position to know certain things about it, depending, of course, on their possession and exercise of recognitional capacities appropriate to the given circumstances. Being thus knowledgeable of a particular endows the subject with a kind of authority. This authority is a power---the subject can possess this authority without exercising it. If in seeing the yellowish red of the tomato I am knowledgeable of that particular, then I am warranted in judging that the tomato is yellowish red whether or not I so judge. But if I do, then, at least in propitious circumstances, I thereby come to know that the tomato is yellowish red. The warrant, here, should be understood as an entitlement to judge (in the ordinary sense of ``entitlement'' and not in Burge's \citeyear{Burge:2003fk} technical sense of the term; compare \citealt[132n]{McDowell:2009ys}). It needn't be conceived as a factor into which knowledge might be analyzed. Moreover, it is an \emph{epistemic} entitlement: The object of my awareness, the yellowish red of the tomato, is a reason that warrants, in the given circumstance, my judging that the tomato is yellowish red where so judging is \emph{coming to know}. Vision confers this epistemic entitlement given the alethic connection between the particular that is the object of visual awareness and the proposition potentially known. Awareness of the sensible truthmaker of a proposition affords the subject with a reason that is in this way akin to proof---it is logically impossible for the particular to exist and the proposition to be false \citep[see][]{Cook-Wilson:1926sf,Kalderon:2010fk}. Because in seeing the yellowish red of the tomato I possess a reason that would, in the given circumstance, warrant my coming to know that the tomato is yellowish red, I am authoritative about the yellowish red of the tomato. My seeing the yellowish red of the tomato can stand proxy for any inquiry on your part about the color of the tomato. If in coming to know that the tomato is yellowish red, I express my knowledge by stating it, I extend to you an offer to take it on my authority that the tomato is the color that I see it to be. 

% section sensory_awareness (end)

\section{Sellars and McDowell on Perceptual Experience} % (fold)
\label{sec:mcdowell_on_sellars_on_perception}

In claiming that vision is a sensory mode of awareness of the mind-independent environment, I am, thus far at least, in agreement with Sellars. Sellars however insists that vision is a mode of \emph{propositional} awareness that is, or constitutes, knowledge of its propositional object. According to Sellars, seeing that something is some way involves undergoing an experience that ``contains a propositional claim''. \citet[§15]{Sellars:1956xp} may have been ``knocking on closed doors'' in 1956, but the thought that perceptual experience has a propositional, or at least an intentional, content is now the prevailing orthodoxy. What distinguishes seeing that something is some way and it merely looking that way is the subject's endorsement of the propositional claim contained in their experience. The endorsement of the propositional claims contained in experience is central to Sellars account of perceptual awareness. It is this feature that brings perception into the space of reasons. However, Sellars held, in addition, that perception essentially involves sensory impressions. 

There are two aspects of sensory impressions as Sellars conceives of them. First, sensory impressions are conscious states of a subject with a phenomenological character. While sensory impressions are conscious states, there is nothing that they are a consciousness \emph{of}. Impressions are nonintentional. Thus Sellars thinks it is a conflation:
\begin{quote}
    \ldots\ to assimilate ``having a sensation of a red traingle'' to ``thinking of a celestial city'' and to attribute to the former \ldots\ the `intentionality' of the latter. \citep[§7]{Sellars:1956xp}
\end{quote}
Sellars is conceiving of impressions on the model of Kantian sensations. According to \citet[B376--7]{Kant:1781fk}, a sensation ``relates to the subject as a modification of its state'' and contrasts with  an intuition in that only the latter is ``immediately related to the object''. On the Kantian model, then, sensations or sensory impressions are nonintentional, conscious states of a subject. Second, the impressions involved in perceptual awareness are the effects of the distal environment on the conscious subject. Seeing that something is some way involves a reliable differential responsiveness to objects, events, and features of the mind-independent environment or facts about these. This reliable differential responsiveness determines no strict correlation. The relationship between distal causes and modifications of the conscious subject is many--one---the same impression involved in seeing that something is the case would be involved in something merely looking to be the case. This reliable differential responsiveness is a species of sensitivity, more generally. In \emph{Science and Metaphysics}, \citet[4]{Sellars:1967uq} describes it as ``sheer receptivity''.

Why does sensory awareness involve impressions? Sellars considers two arguments that correspond to the two aspects of his conception of impressions. In \emph{Empiricism and the Philosophy of Mind}, Sellars argues that the postulation of impressions arises ``in the attempt to explain the facts of sense perception in a scientific style'':
\begin{quote}
    How does it happen that people can have the experience they describe by saying ``It is as though I were seeing a red and triangular physical object'' when either there is no physical object there at all, or, if there it , it is neither red nor triangular? The explanation, roughly, posits that in every case in which a person has an experience of this kind, whether veridical or not, he has what is called a `sensation' or `impression' `of a red triangle'.  \citep[§7]{Sellars:1956xp} % The core idea is that the proximate cause of such a sensation is \emph{only for the most part} brought about by the presence in the neighborhood of the perceiver of a red and triangular physical object; and that while a baby, say, can have the `sensation of a red triangle' without either \emph{seeing} or \emph{seeming to see that the facing side of a physical object is red and triangular}; there usually \emph{looks}, to adults, \emph{to be} a physical object with a red and triangular facing surface, when they are caused to have a `sensation of a red triangle'; while \emph{without} such a sensation, no such experience can be had. \citep[§7]{Sellars:1956xp}
\end{quote}
While \cite[§7]{Sellars:1956xp} considers the explanation ``in a scientific style'', he does not directly endorse it, commenting that he will ``have a great deal more to say about this kind of `explanation' of perceptual situations in the course of my argument''. Indeed, this may be, at least in part, a dialectical concession to the sense datum theory corresponding, as it does, to the generalizing step of the argument from illusion. As I read \emph{Empiricism and the Philosophy of Mind}, Sellars' \citeyearpar[§7]{Sellars:1956xp} argument against the sense data theory does not rest with the the charge that ``the classical concept of a sense datum'' is ``a mongrel resulting from the crossbreeding of two ideas''---a nonclassical form of the sense data theory may yet be free of the mongrel conflation. In §10, Sellars claims that ``a reasonable next step would be to examine these two ideas and determine how that which survives criticism in each is properly to be combined with the other.'' This task will occupy him for the rest of the work, and the sense datum theory is not completely dispensed with until the penultimate paragraph. Since it is unclear whether and to what extent Sellars endorses the explanation ``in a scientific style'', I will set it aside.

While the explanation ``in a scientific style'' focuses on the phenomenological aspect of sensory impressions, the argument that Sellars offers in \emph{Science and Metaphysics} focuses on the other aspect of impressions---the way in which impressions are the potential effects of distal causes:
\begin{quote}
    \ldots\ it is only if Kant distinguishes the radically nonconceptual character of sense from the conceptual character of the synthesis of apprehension in intuition \ldots\ and, accordingly \emph{receptivity} of sense from the \emph{guidedness} of intuition that he can avoid the dialectic which leads from Hegel's \emph{Phenomenology} to nineteenth century idealism. \citep[16]{Sellars:1967uq}
\end{quote}
McDowell reconstructs Sellars' reasoning as follows:
\begin{quote}
    Sellars thinks that conceptual representations in perception must be guided by manifolds of ``sheer receptivity'', because he thinks that only so can we make it intelligible to ourselves that conceptual occurrences in perceptual experience---and thereby ultimately thought, conceptual activity, in general---are constrained by something external to conceptual activity. And as he sees, we need such external constraint in our picture if we are to be entitled to take it that conceptual activity is directed towards an independent reality, as it must be if it is to be intelligible as conceptual activity at all. \citep[46]{McDowell:1998vn}
\end{quote}

However, McDowell maintains that if Sellars has available a particular conception of perceptual experience, as the conceptual shaping of sensory consciousness, then, while perceptual experience must be guided from without, there is no need for manifolds of ``sheer receptivity'' to play this guiding role:
\begin{quote}
    But I suggested that once we understand how objects can be immediately present to conceptually shaped sensory consciousness in intuition, we can take this need for external constraint to be met by perceived objects themselves. \citep[46]{McDowell:1998vn}
\end{quote}
\begin{quote}
    In a way we are now equipped to understand, \ldots\ the guidance is supplied by the \emph{objects} themselves the subject matter of those conceptual representations, becoming immediately present to sensory consciousness of the subjects of these conceptual goings-on. \citep[467]{McDowell:1998vn}
\end{quote}

This is an interesting and ultimately cogent objection to the view that McDowell attributes to Sellars. However, McDowell's presentation of it obscures the true nature of the objection.

% \begin{quote}
%     Sellars's idea is that for thought to be intelligibly of objective reality, the conceptual representations involved in perceptual experience must be guided from without. And indeed they are, I can say. But there is no need for manifolds of ``sheer receptivity'' to play this guiding role. In a way we are now equipped to understand, given the conception of intuitions adumbrated in the ``Clue'', the guidance is supplied by the \emph{objects} themselves the subject matter of those conceptual representations, becoming immediately present to sensory consciousness of the subjects of these conceptual goings-on. \citep[467]{McDowell:1998vn}
% \end{quote}

To see this, consider the following \emph{tu quo que} that a naïve realist might press against McDowell: 
\begin{quote}
    Perceptual experiences must be guided from without. And indeed they are, I can say. But there is no need for conceptual representations to play this guiding role. In a way that we are now equipped to understand, given the conception of perception adumbrated in the previous section, the guidance is supplied by the objects becoming immediately present to sensory consciousness. Once we understand how objects can be immediately present to sensory consciousness, we can take this need for external constraint to be met by perceived objects themselves, irrespective of any conceptual goings-on.
\end{quote}
Regardless of the ultimate dialectical effectiveness of this response, I believe that reflection on the mere availability of the \emph{tu quo que} reveals distinguishable elements in McDowell's objection to Sellars. Specifically, we can distinguish negative and positive claims here.

% section mcdowell_on_sellars_on_perception (end)

\section{What Myth?} % (fold)
\label{sec:what_myth_}

What remains is a substantive disagreement that arises at a fine-grained level of theoretical description. It would be disingenuous to claim, in advance of further inquiry, that either party to this dispute has fallen prey to some antecedently understood myth.

% section what_myth_ (end)

% Bibligography
\bibliographystyle{plainnat} 
\bibliography{Philosophy} 

\end{document}