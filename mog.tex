%!TEX TS-program = xelatex 
%!TEX TS-options = -synctex=1 -output-driver="xdvipdfmx -q -E"
%!TEX encoding = UTF-8 Unicode
%
%  mog
%
%  Created by Mark Eli Kalderon on 2010-04-06.
%  Copyright (c) 2010. All rights reserved.
%

\documentclass[12pt]{article} 

% Definitions
\newcommand\mykeywords{given, Sellars, perception, reasons} 
\newcommand\myauthor{Mark Eli Kalderon} 
\newcommand\mytitle{Before the Law}

\include{preamble/preamble.tex}

%%% BEGIN DOCUMENT
\begin{document}

% Title Page
\maketitle
% \begin{abstract} % optional
% \noindent
% \end{abstract} 
\vskip 2em \hrule height 0.4pt \vskip 2em
% \epigraph{text of epigraph}{\textsc{author of epigraph}} % optional; make sure to uncomment \usepackage{epigraph}

% Layout Settings
\setlength{\parindent}{1em}

% Main Content
\epigraph{Before the law sits a gatekeeper. To this gatekeeper comes a man from the country who asks to gain entry into the law. But the gatekeeper says that he cannot grant him entry at the moment. The man thinks about it and then asks if he will be allowed to come in sometime later on. “It is possible,” says the gatekeeper, “but not now.”}

\section{Introduction} % (fold)
\label{sec:introduction}
It is notoriously difficult to understand the distinctive charge of falling prey to the Myth of the Given. The principal obstacle to understanding the nature of the charge was erected by Sellars himself. The Myth of the Given is said to arise in various forms and yet each form participates in a framework of givenness---a framework that Sellars declines to specify in general terms. As interpreters of Sellars we may speculate about the framework of givenness. It is reasonable to at least attempt to reconstruct the framework of givenness in light of Sellars' discussion of the more specific forms of the Myth of the Given and especially in light of the conceptual genealogies he offers. However there is no consensus on how the Myth of the Given is to be understood in general terms even among interpreters philosophically sympathetic with Sellars on key issues. Thus different accounts of the Myth are given by Brandom, McDowell, Rorty, and Williams.

Suppose one accepts a view about perception that makes one liable to the charge of having fallen prey to the Myth of the Given. One can know or at least judge with relative certainty that a view of perception is a form of the Myth of the Given (if it is indeed a myth) without knowing in general terms what the Myth of the Given is. So, for example, I am increasingly attracted to a view of visual perception where vision affords the perceiver with a sensory mode of awareness of particulars spatially distant from the perceiver's body. This distinctive sensory mode of awareness is non-propositional---it doesn't take a fact as its object, but a particular. Moreover this sensory mode of awareness is fundamental and irreducible. This fundamental and irreducible sensory mode of awareness of spatiotemporal particulars arrayed in a mind-independent environment makes one knowledgeable of that environment. In being so aware of a particular, the perceiver is in a position to know certain things about it, depending, of course of the perceiver's possession and exercise of recognitional capacities appropriate to the given circumstances. We can be confident that Sellars at least would regard this as a form of the Myth of the Given since it involves a pre-conceptual mode of awareness that makes propositional knowledge available to the perceiving subject, whether or not such knowledge is activated.


% section introduction (end)

% Bibligography
\bibliographystyle{plainnat} 
\bibliography{Philosophy} 

\end{document}