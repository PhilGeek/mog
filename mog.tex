%!TEX TS-program = xelatex 
%!TEX TS-options = -synctex=1 -output-driver="xdvipdfmx -q -E"
%!TEX encoding = UTF-8 Unicode
%
%  mog
%
%  Created by Mark Eli Kalderon on 2010-04-06.
%  Copyright (c) 2010. All rights reserved.
%

\documentclass[12pt]{article} 

% Definitions
\newcommand\mykeywords{given, Sellars, perception, reasons} 
\newcommand\myauthor{Mark Eli Kalderon} 
\newcommand\mytitle{Before the Law}

\input{preamble/preamble.tex}

%%% BEGIN DOCUMENT
\begin{document}

% Title Page
\maketitle
% \begin{abstract} % optional
% \noindent
% \end{abstract} 
\vskip 2em \hrule height 0.4pt \vskip 2em
% \epigraph{text of epigraph}{\textsc{author of epigraph}} % optional; make sure to uncomment \usepackage{epigraph}

% Layout Settings
\setlength{\parindent}{1em}

% Main Content
\epigraph{Before the law sits a gatekeeper. To this gatekeeper comes a man from the country who asks to gain entry into the law. But the gatekeeper says that he cannot grant him entry at the moment. The man thinks about it and then asks if he will be allowed to come in sometime later on. “It is possible,” says the gatekeeper, “but not now.”---Franz Kafka}

\section{Introduction} % (fold)
\label{sec:introduction}
It is notoriously difficult to understand the distinctive charge of falling prey to the Myth of the Given. The principal obstacle was erected by Sellars himself. The Myth of the Given is said to arise in various forms and yet each of these forms participates in a framework of givenness---a framework that Sellars declines to specify in general terms. As interpreters of Sellars, we may speculate about how best to understand the general framework of givenness. It is reasonable to attempt to reconstruct that framework in light of Sellars' discussion of the more specific forms of the Myth and especially in light of the conceptual genealogies he offers. However, there is no consensus on how the Myth of the Given is to be understood in general terms even among interpreters philosophically sympathetic with Sellars on key issues. Thus different accounts of the Myth are given by \citet{Brandom:2002fk}, \citet{Coates:2007ly}, \citet{deVries:2005qf}, \citet{McDowell:1996uq,McDowell:1998vn}, \citet{Rosenberg:2007ve}, \citet{Rorty:1979ys}, and \citet{Williams:1977zr}.

One can know or at least judge with relative certainty that a view of perception is a form of the Myth of the Given (if it is indeed a myth) without knowing in general terms what the Myth of the Given is. So, for example, I am increasingly attracted to a view of visual perception where vision affords the perceiver with a sensory mode of awareness of particulars spatially distant from the perceiver's body. This distinctive sensory mode of awareness is non-propositional---it doesn't take a fact as its object, but a particular. Moreover this sensory mode of awareness is fundamental and irreducible. This fundamental and irreducible sensory mode of awareness of spatiotemporal particulars arrayed in a mind-independent environment makes one knowledgeable of that environment. In being so aware of a particular, the perceiver is in a position to know certain things about it, depending, of course of the perceiver's possession and exercise of recognitional capacities appropriate to the given circumstances. We can be confident that Sellars, at least, would regard this as a form of the Myth of the Given since it involves a pre-conceptual mode of awareness that makes propositional knowledge available to the perceiving subject, whether or not such knowledge is activated (in Williamson's \citeyear{Williamson:1990uq} terminology.) 

Suppose, then, one accepts a view about perception that makes one liable to the charge of falling prey to the Myth of the Given. How might one respond to the charge? Not directly, it seems. For suppose one could argue that the Myth of the Given was no myth on a specific interpretation of that charge. Given the lack of consensus on how the framework of givenness is to be understood in general terms, it would remain an open question whether the view defended was yet another form of the Myth (on some other interpretation of it). A lack of a definitive statement of the Myth from Sellars (and a lack of consensus on how best to reconstruct the general framework of givenness) renders the charge elusive and Protean. Being liable to the charge is like being the man from the country who seeks access to the law in Kafka's parable. Denied access to law, should one simply give in to despair before its gatekeepers?

While no direct response is possible, perhaps an \emph{indirect} response is. I can think of two complementary strategies. First, one might strive to satisfy oneself that perception, so conceived, is epistemically significant---that despite the charge of falling prey to the Myth of the Given, the account makes it intelligible that perception makes one knowledgeable of a mind-independent subject matter. Of course, by itself, this would fail to persuade anyone inclined to prosecute that charge, but it is an essential first step that promises to clarify what's at issue in the dispute. Second, one might examine rival accounts of perception tailor made to avoid the Myth of the Given such as the accounts provided by \citet{Sellars:1956xp} and \citet{McDowell:1996uq}. If criticism of accounts designed specifically to avoid the Myth naturally motivates one's preferred account, this is some reason, at least, to think that Myth of the Given is no myth. 

These, then, are the strategies that I will be pursuing in this essay. In the first part, I will sketch how perception, conceived as a non-propositional sensory mode of awareness, can make the subject knowledgeable of a world without the mind. And in the second part, I will discuss Sellars and McDowell on perceptual experience. Of particular interest will be McDowell's criticism that Sellars misconceives the nature of sensibility by allowing receptivity to operate independently of the subject's conceptual capacities. There are distinguishable claims made in the course of levelling this criticism. Marking this distinction allows us to see that McDowell is right in rejecting Sellars' account, but wrong in thinking that this requires our conceptual capacities to be operative in perceptual awareness. 

\section{Sensory Awareness} % (fold)
\label{sec:sensory_awareness}

% section sensory_awareness (end)

\section{McDowell and Sellars on Perceptual Experience} % (fold)
\label{sec:mcdowell_on_sellars_on_perception}

% section mcdowell_on_sellars_on_perception (end)

% section introduction (end)

% Bibligography
\bibliographystyle{plainnat} 
\bibliography{Philosophy} 

\end{document}