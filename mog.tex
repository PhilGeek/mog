%!TEX TS-program = xelatex 
%!TEX TS-options = -synctex=1 -output-driver="xdvipdfmx -q -E"
%!TEX encoding = UTF-8 Unicode
%
%  mog
%
%  Created by Mark Eli Kalderon on 2010-04-06.
%  Copyright (c) 2010. All rights reserved.
%

\documentclass[12pt]{article} 

% Definitions
\newcommand\mykeywords{given, Sellars, perception, reasons} 
\newcommand\myauthor{Mark Eli Kalderon} 
\newcommand\mytitle{Before the Law}

\input{preamble/preamble.tex}

%%% BEGIN DOCUMENT
\begin{document}

% Title Page
\maketitle
% \begin{abstract} % optional
% \noindent
% \end{abstract} 
\vskip 2em \hrule height 0.4pt \vskip 2em
% \epigraph{text of epigraph}{\textsc{author of epigraph}} % optional; make sure to uncomment \usepackage{epigraph}

% Layout Settings
\setlength{\parindent}{1em}

% Main Content
\epigraph{Before the law sits a gatekeeper. To this gatekeeper comes a man from the country who asks to gain entry into the law. But the gatekeeper says that he cannot grant him entry at the moment. The man thinks about it and then asks if he will be allowed to come in sometime later on. “It is possible,” says the gatekeeper, “but not now.”---Franz Kafka}

\section{Introduction} % (fold)
\label{sec:introduction}
It is notoriously difficult to understand the distinctive charge of falling prey to the Myth of the Given. The principal obstacle was erected by Sellars himself. The Myth of the Given is said to arise in various forms, and yet each of these forms participates in a framework of givenness---a framework that Sellars declines to specify in general terms. As interpreters of Sellars, we may speculate about how best to understand the general framework of givenness. It is reasonable to attempt to reconstruct that framework in light of Sellars' discussion of the more specific forms of the Myth and especially in light of the conceptual genealogies he offers. However, there is no consensus on how the Myth of the Given is to be understood in general terms even among philosophers sympathetic with Sellars on fundamental issues. Thus different accounts of the Myth are given by \citet{Brandom:2002fk}, \citet{Coates:2007ly}, \citet{deVries:2005qf}, \citet{McDowell:1996uq,McDowell:1998vn}, \citet{Rosenberg:2007ve}, \citet{Rorty:1979ys}, and \citet{Williams:1977zr}.

One can know, or at least reasonably judge, that a conception of perception is a form of the Myth of the Given (if it is indeed a myth) without knowing in general terms what the Myth of the Given is. So, for example, I am increasingly attracted to a conception of visual perception where vision affords the perceiver with a sensory mode of awareness of particulars spatially distant from the perceiver's body. This distinctive sensory mode of awareness is non-propositional---\-it doesn't take a fact as its object, but a particular. Moreover this sensory mode of awareness is fundamental and irreducible. This fundamental and irreducible sensory mode of awareness of spatiotemporal particulars arrayed in a mind-independent environment makes one knowledgeable of that environment. In being so aware of a particular, the perceiver is in a position to know certain things about it, depending, of course, on the perceiver's possession and exercise of recognitional capacities appropriate to the given circumstances. We can be confident that Sellars, at least, would regard this as a form of the Myth of the Given since it involves a pre-conceptual mode of awareness that makes propositional knowledge available to the perceiving subject, whether or not such knowledge is in fact activated (in Williamson's \citeyear{Williamson:1990uq} terminology.) 

Suppose, then, one is liable to the charge of falling prey to the Myth of the Given. How might one respond to the charge? Not directly, it seems. For suppose one could argue that the Myth of the Given was no myth on a specific understanding of that charge. Given the lack of consensus on how the framework of givenness is to be understood, it would remain an open question whether the conception defended was yet another form of the Myth on a distinct and potentially superior understanding of it. A lack of a definitive statement of the Myth from Sellars and a lack of consensus on how best to reconstruct the general framework of givenness renders the charge elusive and Protean. Being liable to the charge is like being the man from the country who seeks access to the law in Kafka's parable. Denied access to law, should one simply despair before its gatekeepers?

While no direct response is possible, perhaps an \emph{indirect} response is. I can think of two complementary strategies. First, one might strive to satisfy oneself that perception, so conceived, is epistemically significant---that the conception renders intelligible that perception should make one knowledgeable of a mind-independent subject matter. Of course, by itself, this would fail to persuade anyone inclined to prosecute that charge, but persuading the prosecution is unnecessary to achieve the modest task of setting one's own mind at ease about the epistemic significance of perception as one conceives it to be. The task may be modest in this way, but it is an essential first step that promises to clarify what's at issue in the dispute. Second, one might examine rival conceptions of perception that are tailor-made to avoid the Myth of the Given such as those provided by \citet{Sellars:1956xp} and \citet{McDowell:1996uq,McDowell:1998vn,McDowell:2008fk}. If criticism of conceptions designed specifically to avoid the Myth naturally motivates a conception that is known, or reasonably judged, to be a form of the Myth, this is some reason, at least, to think that Myth of the Given is no myth. 

These, then, are the strategies that I will be pursuing. In the first part, I proceed dogmatically, sketching how visual perception, conceived as a non-propositional sensory mode of awareness, can make the subject knowledgeable of particulars without the mind. Doing so will clarify at least one aspect of the issue in dispute---one source of disagreement about the epistemic significance of perception, so conceived, turns on a disagreement about the metaphysics of reasons. In the second part, I proceed dialectically, discussing Sellars and McDowell on perceptual experience. Of particular interest will be McDowell's criticism that Sellars misconceives the nature of sensibility by allowing receptivity to operate independently of the subject's conceptual capacities. There are distinguishable claims made in the course of leveling this criticism. Marking these allows us to see that McDowell is right in rejecting Sellars' account, but wrong in thinking that this requires our conceptual capacities to be operative in perceptual awareness. % This involves thinking of the awareness of spatially distant particulars that vision affords the perceiver as pre-conceptual---as a form of the Myth. But if it is, the Myth is no myth. Or so I will argue.

\section{Sensory Awareness} % (fold)
\label{sec:sensory_awareness}

Dissatisfaction with the quality of tomatoes available in central London and not a concern for the veracity of philosophical clichès was my reason for growing tomatoes by my window. It is a large plant in a terra cotta pot that sits on a wooden stool set against a generous Victorian window. In the background is a field of rubble where Middlesex Hospital used to be. Everything has been leveled except for the chapel. Looking up I see a ripening tomato set against this complex scene. It is not quite ripe---it is a yellowish red, if not orange. Moreover, it is dappled in sunlight that has just emerged from the clouds in the aftermath of a shower. When I see the ripening tomato, the ripening tomato is the object of my perceptual experience---I am aware of the tomato in my perceptual experience of it. 

The awareness involved in visual perception is \emph{sensory} in the way that the awareness of a passing thought in the stream of consciousness is not. Moreover, vision affords the subject with a distinctive \emph{mode} of sensory awareness. Vision is not alone among the senses in providing information about the distal environment. Thus we can \emph{see} the leaves rustling and \emph{hear} them rustling. Perhaps, as Berkeley urged, we hear, at least in the first instance, the sound of the rustling leaves. But, if we do, then, at least in propitious circumstances, we hear the source of the sound by hearing the sound. So whereas vision affords the subject with a \emph{visual} mode of awareness of the event, arguably at least, audition affords the subject with an \emph{auditory} mode of awareness of the very same event. (For recent discussion of audition see Nudds and O'Callaghan \citeyear{Nudds:2009sk}).

The objects of visual awareness are spatiotemporal particulars arrayed in a mind-independent environment. These particulars do not constitute a unified ontological category. Among them are objects (the plant, the stool), events (the rustling of the leaves, the sun's setting), and property instances (the yellowish red of the tomato). But objects, events, and property instances have distinct modes of being. Thus, for example, events are spatiotemporal particulars that unfold through time in a way that neither objects nor property instances do (though see Sider \citeyear{Sider:1997fk} for criticism of this way of distinguishing events from objects; see also Hawthorne \citeyear{Hawthorne:2008uq} and Fine \citeyear{Fine:2006fk} for further discussion). The particulars that we see also differ in degree of both substantiality and being. In addition to ordinary material substances like tomatoes, we see flashes, flames, reflections, and rainbows, on the one hand, and shadows and holes, on the other hand. Flashes are unusual events. They are colored, but most events are not colored despited having colored participants. (What color was the Battle of Kosovo?) But what is presently important is that they also lack substrata. And the plant's shadow is constituted by a relative decrease in the amount of light uniformly distributed across the visible spectrum in a region determined by the positions of the plant and the light source. But a relative decrease is a privation or diminution of being. Shadows may be creatures of darkness, but that does not make them candidates for elimination. \citet{Quine:1948ef} may crack wise about the possible fat man in the door way, but perception makes shadows, a species of privation, the subject matter of demonstrative thought and talk. Despite being particulars, the objects of perception differ, in this way, in mode, substantiality, and being. (On the heterogeneity of the objects of visual perception see Austin \citeyear{Austin:1962lr}.) % Moreover, perception makes available demonstrative thoughts about these particulars. Seeing the yellow-red tomato, I recognize it's state of maturation and think---that's not ripe yet. But perception also makes available demonstrative thoughts about flashes---that's red---and holes---that's too big to jump across. 

Vision makes spatially distant particulars visually present. When I see the yellowish red tomato, I see a particular spatially distant from me. Moreover, I experience the yellowish red as inhering in the external surface of that particular. Hering's \citeyearpar[8]{Hering:1920ty} ringed shadow experiment demonstrates the phenomenological difference between a color merely overlaying a surface and that color inhering in that surface. When a shadow is cast on a white surface you see the shadow as ``an incidental darkness that lies on the [surface]''. If now a black line is drawn around the shadow completely obscuring its penumbra, the darkness no longer appears to lie on the surface, but rather appears to inhere in the surface. Only then does the shadowed white surface misleadingly look to be gray. In seeing the yellowish red tomato, I experience the yellowish red as inhering in the external surface of the tomato. Of course, not all colors inhere in substances or their parts. A flash can be blue even though the flash lacks a substrata in which the blue could inhere. But even if the blue of the flash inheres in nothing, I experience the blue in the remote spatiotemporal region of the flash. 

Thus, Broad writes:
\begin{quote}
    In its purely phenomenological aspect \emph{seeing} is ostensibly \emph{saltatory}. It seems to leap the spatial gap between the percipient's body and a remote region of space. Then again, it is ostensibly \emph{prehensive} of the surfaces of distant bodies as coloured and extended, and of external events as colour-occurences \emph{localized} in remote regions of space. \citep[32]{Broad:1965dq}
\end{quote}
Seeing is saltatory in that the objects of visual awareness are spatially remote; seeing is prehensive in that its objects are present in our visual awareness of them. ``Prehensive'' belongs to a primordial family of broadly tactile metaphors for visual awareness that includes ``grasping'', and ``apprehending''. What unites these metaphors is that they are all a mode of taking in, and ``ingestion'' is a natural variant \citep[see][]{Johnston:2006uq}. The tactile nature of these metaphors can mislead, however, if we take too seriously the contact involved in taking in an object. Thus Broad remarks that ``It is a natural, if paradoxical, way of speaking to say that seeing seems to `bring us into \emph{contact} with \emph{remote} objects' and to reveal their shapes and colors''. The air of paradox, however, is removed once we recognize that the sense in which visual awareness brings us into contact with particulars does not conflict with the sense in which these particulars are remote. To be sure, seeing the tomato a meter away does not make the tomato proximate, but that just means that the sense in which the tomato is present in perceptual awareness is not the same sense as the tomato being spatially present.

The phenomenology of being visually presented with colors inhering in external bodies played a role in Russell's felt exuberance at abandoning idealism:
\begin{quote}
	I felt it, in fact, as a great liberation, as if I had escaped from a hothouse on to a wind-swept headland. I hated the stuffiness involved in supposing that space and time were only in my mind. I liked the starry heavens even better than the moral law, and could not bear Kant’s view that the one I liked best was only a subjective figment. In the first exuberance of liberation, I became a naïve realist and rejoiced in the thought that grass is really green, in spite of the adverse opinions of all philosophers from Locke onwards. \citep[48]{Russell:1959fv}
\end{quote}
What Russell rejoiced in was warrantedly taking external bodies to be the way his perceptual experience presents them to be---as having colors inhering in the opaque surfaces they present. He rejoiced in the greenness of the grass revealed by sight. Compare Melville's revulsion at Locke's metaphysics:
\begin{quote}
	And when we consider that other theory of the natural philosophers, that all other earthly hues---every stately or lovely emblazoning---the sweet tinges of sunset skies and woods; yea, and the gilded velvets of butterflies, and the butterfly cheeks of young girls; all these are but subtile deceits, not actually inherent in substances, but only laid on from without; so that all deified Nature absolutely paints like the harlot, whose allurements cover nothing but the charnel-house within \ldots\ \citep[ch. 42]{Melville:1851ms}
\end{quote}
Freed from the adverse opinions of all philosophers from Locke onwards, one need no longer think that Nature paints like a harlot and can once again rejoice in the lovely emblazoning of an English pastoral scene.

Visual perception involves a sensory mode of awareness that takes external particulars as objects. This is a claim about sense perception, not sense experience (understood as the genus of which sense perception is a species). It echoes a conception of perception common among early twentieth century realists. There are, of course, important differences in detail between the present conception and its early twentieth century precursors. Thus, for example, Russell held that we are acquainted with universals as well as particulars. And the Cambridge realists held that the distinctive sensory mode of awareness was operative not just in sense perception but in sense experience more generally. They were thus committed to a kind of \emph{experiential monism} (in Snowdon's \citeyear{Snowdon:2008oz} terminology). Specifically, all sense experience was conceived to involve, as part of its nature, a sensory mode of awareness. According to the Cambridge realists, then, even subject to illusion or hallucination, there is something of which one is aware. But with that, they were an application of the argument from illusion, or hallucination, or conflicting appearances away from immaterial sense data and a representative realism that tended, over time, to devolve into a form of phenomenalism. 

Sellars observes that if sensory awareness takes particulars as its object, then it cannot be a form of propositional knowledge:
\begin{quote}
	[W]e may well experience a feeling of surprise on noting that according to sense-datum theorists, it is \emph{particulars} that are sensed. For what is \emph{known}, even in in non-inferential knowledge, is \emph{facts} rather than particulars, items of the form \emph{something's being thus and so} or \emph{something's standing in a certain relation to something else}. It would seem, then, that the sensing of sense contents \emph{cannot} constitute knowledge, inferential \emph{or} non-inferential \ldots\ \citep[§3]{Sellars:1956xp}
\end{quote}
What's known are thoughts, propositions, facts---not particulars. Sellars hasn't explicitly given us a reason for distinguishing thoughts and particulars in this way. Perhaps he regards the distinction as evident. Prichard, however, gives us a reason, indeed, the right kind of reason. According to Prichard, thoughts have a kind of generality that precludes them from being particulars:
\begin{quote}
	There seems to be no way of distinguishing perception and conception as the apprehension of different realities except as the apprehension of the individual and the universal respectively. Distinguished in this way, the faculty of perception is that in virtue of which we apprehend the individual, and the faculty of conception is that power of reflection in virtue of which a universal is made the explicit object of thought. \citep[]{Prichard:1909yg}
\end{quote}
(For contemporary discussion of particularity and the content of perception see Brewer \citeyear{Brewer:2008fk}, Martin, \citeyear{Martin:2002jb} Soteriou \citeyear{Soteriou:2000iz,Soteriou:2005fk}, and Travis \citeyear{Travis:2005ys}.)

Sellars' observation is the basis of the following dilemma:
\begin{quote}
    The sense-datum theorist, it would seem, must choose between saying:
    \begin{enumerate}
        \item[(a)] It is \emph{particulars} which are sensed. Sensing is not knowing. The existence of sense-data does not \emph{logically} imply the existence of knowledge.
        \item[(b)] Sensing \emph{is} a form of knowing. It is \emph{facts} rather than \emph{particulars} which are sensed. \citep[§3]{Sellars:1956xp}
    \end{enumerate}
\end{quote}
Notice so far in Sellars' discussion all that is in play of the sense-datum theory is the act--object structure of sensing and the hypothesis that the objects of sensings are particulars. Both claims can and have been endorsed by philosophers who are not themselves sense-datum theorists---including naïve realists, disjunctivists, and advocates of the theory of appearing. So Sellars' dilemma for the sense-datum theorist is applicable to other positions as well, including the present conception.

If visual awareness takes particulars as objects, it is not a form of propositional knowledge. But that does not mean that vision, so conceived, lacks epistemic significance. Vision can be a source of knowledge insofar as the perceiving subject can recognize the object of perception for what it is. When I look at the ripening tomato, the tomato is present in my awareness of it. Moreover, if I possess recognitional capacities appropriate to the given circumstances, in being so aware of the ripening tomato, I can come to know various things about it---that it is yellowish red, say. The non-propositional sensory mode of awareness involved in visual perception may not be a form of knowledge, but it makes me \emph{knowledgeable} of its object. Sensory awareness makes the subject knowledgeable of its object in the sense that it makes knowledge about its object \emph{available} to the perceiving subject, depending, of course, on the subject's possession and exercise of recognitional capacities appropriate to the given circumstances. Perception constitutes a change in the subject's knowledge potential whether or not such knowledge is in fact activated.

In looking at the ripening tomato by my window, I see the yellowish red of the tomato. If I recognize what I thus see, I can come to know that the tomato is yellowish red. What I know is a proposition---that the tomato is yellowish red. What I see is a particular---the yellowish red of the tomato. How can awareness of that particular make available knowledge of that proposition? The yellowish red of the tomato is a \emph{truthmaker} of the proposition that the tomato is yellowish red \citep[see][]{Johnston:2006uq}. If a particular, \( x \)---be it an object, event, or property instance---is a truthmaker for a proposition, \( p \), then it is metaphysically necessary that if \( x \) exists, then \( p \) is true. The yellowish red of the tomato exist; so, necessarily, the proposition that the tomato is yellowish red is true. Vision makes me knowledgeable of particulars arrayed in a mind-independent environment by making me aware of sensible truthmakers.

A residual doubt may remain. Sellars famously connects knowledge and reasons in the following fashion:
\begin{quote}
	The essential point is that in characterizing an episode or state as that of \emph{knowing}, we are not giving an empirical description of that episode or state; we are placing it in the logical space of reasons, of justifying and being able to justify what one says. \citep[§36]{Sellars:1956xp}
\end{quote}


% section sensory_awareness (end)

\section{McDowell and Sellars on Perceptual Experience} % (fold)
\label{sec:mcdowell_on_sellars_on_perception}

% section mcdowell_on_sellars_on_perception (end)

% section introduction (end)

% Bibligography
\bibliographystyle{plainnat} 
\bibliography{Philosophy} 

\end{document}