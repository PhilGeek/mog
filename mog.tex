%!TEX TS-program = xelatex
%!TEX TS-options = -synctex=1
%!TEX encoding = UTF-8 Unicode
%
%  mog
%
%  Created by Mark Eli Kalderon on 2010-04-06.
%  Copyright (c) 2010. All rights reserved.
%

\documentclass[12pt]{article} 

% Definitions
\newcommand\mykeywords{given, Sellars, perception, reasons} 
\newcommand\myauthor{Mark Eli Kalderon} 
\newcommand\mytitle{Before the Law}
\newcommand{\change}{\textcolor{blue}{\textbf{CHANGE SLIDE}}}

% Packages
\usepackage{geometry} \geometry{a4paper}
\usepackage{color}

% XeTeX
\usepackage{fontspec}
\usepackage{xltxtra,xunicode}
\defaultfontfeatures{Scale=MatchLowercase,Mapping=tex-text}
\setmainfont{Hoefler Text}

% Section Formatting
\usepackage[]{titlesec}
\titleformat{\section}[hang]{\fontsize{14}{14}\scshape}{\S{\thesection}}{.5em}{}{}
\titleformat{\subsection}[hang]{\fontsize{12}{12}\scshape}{\S{\thesubsection}}{.5em}{}{}
\titleformat{\subsubsection}[hang]{\fontsize{12}{12}\scshape}{\S{\thesubsubsection}}{.5em}{}{}


% Title Information
\title{\mytitle}% 
\author{\myauthor} 
\date{} % Leave blank for no date, comment out for most recent date

% PDF Stuff
\usepackage[plainpages=false, pdfpagelabels, bookmarksnumbered, backref, pdftitle={\mytitle}, pagebackref, pdfauthor={\myauthor}, pdfkeywords={\mykeywords}, xetex, dvipdfmx]{hyperref} 

%%% BEGIN DOCUMENT
\begin{document}

% Title Page
\maketitle

% Main Content

% Layout Settings
\setlength{\parindent}{1em}

\change

My title is a reference to a parable by Kafka. It's significance will become clear. It is, however, useful to have that parable fresh before your mind. And so without further ado, ladies and gentleman, I give you Mr Orson Welles \ldots\ 

\change

\noindent It is notoriously difficult to understand the charge of falling prey to the Myth of the Given. The principal obstacle was erected by Sellars himself. The Myth of the Given is said to arise in various forms, and yet each of these forms participates in a framework of givenness---a framework that Sellars declines to specify in general terms. As interpreters of Sellars, we may speculate about how best to understand the general framework of givenness. It is reasonable to attempt to reconstruct that framework in light of Sellars' discussion of the more specific forms of the Myth and especially in light of the conceptual genealogies he offers. However, there is no consensus on how the Myth of the Given is to be understood in general terms even among philosophers sympathetic with Sellars on fundamental issues.

Nevertheless, there may be occasions where one can know, or at least reasonably judge, that a conception of perception is a form of the Myth (if it is indeed a myth) without knowing in general terms what the Myth of the Given is. So, for example, I am increasingly attracted to a conception of visual perception as a mode of taking in---as affording the perceiver with a sensory mode of awareness of particulars spatially distant from the perceiver's body. This distinctive sensory mode of awareness is not propositional---it does not take a fact as its object, but a particular. Nor is it in any way the actualization of the subject's conceptual capacities. This fundamental and irreducible sensory mode of awareness of spatiotemporal particulars arrayed in a mind-independent environment makes one knowledgeable of that environment. In being so aware of a particular, the perceiver is in a position to know certain things about it, depending, of course, on their possession of recognitional capacities appropriate to the given circumstances. We can be confident that Sellars, at least, would regard this as a form of the Myth since it involves a preconceptual mode of awareness.

Suppose, then, one is liable to the charge of falling prey to the Myth of the Given. How might one respond to the charge? Not directly, it seems. For suppose one could argue that the Myth of the Given was no myth on a specific understanding of that charge. Given the lack of consensus on how the framework of givenness is to be understood, it would remain an open question whether the conception defended was yet another form of the Myth on a distinct and potentially superior understanding of it. A lack of a definitive statement of the Myth from Sellars and a lack of consensus on how best to reconstruct the general framework of givenness renders the charge elusive and Protean. Being liable to the charge is like being the man from the country who seeks access to the law in Kafka's parable. Denied access to law, should one simply despair before its gatekeepers?

While no direct response is possible, perhaps an \emph{indirect} response is. I can think of two complementary strategies. First, one might strive to satisfy oneself that perception, so conceived, is epistemically significant---that the conception renders intelligible that perception should make one knowledgeable of a mind-independent subject matter. Of course, by itself, this would fail to persuade anyone inclined to prosecute that charge, but persuading the prosecution is unnecessary to achieve the modest task of setting one's own mind at ease about the epistemic significance of perception as one conceives it to be. Second, one might examine rival conceptions of perception that are tailor-made to avoid the Myth of the Given such as those provided by Sellars and McDowell. If criticism of conceptions designed specifically to avoid the Myth naturally motivates a conception that is known, or reasonably judged, to be a form of the Myth, this is some reason, at least, to think that the Myth of the Given is no myth.

These, then, are the strategies that I will be pursuing.

\change

Dissatisfaction with the quality of tomatoes available in central London and not a concern for the veracity of philosophical clichés was my reason for growing tomatoes by my window. It is a large plant in a terra cotta pot that sits on a wooden stool set against a generous Victorian window. In the background is a field of rubble where Middlesex Hospital used to be. Everything has been leveled except for the chapel. Looking up I see a ripening tomato set against this complex scene. It is not quite ripe---it is a yellowish red, if not orange. Moreover, it is dappled in sunlight that has just emerged from the clouds in the aftermath of a shower. When I see the ripening tomato, the ripening tomato is the object of my perceptual experience---I am aware of the tomato in my perceptual experience of it. 

The objects of visual awareness are spatiotemporal particulars arrayed in a mind-independent environment. These particulars do not constitute a unified ontological category. Among them are objects (the plant, the stool), events (the rustling of the leaves, the sun's setting), and property instances (the yellowish red of the tomato). But objects, events, and property instances have distinct modes of being. Thus, for example, events unfold through time in a way that neither objects nor property instances do. The particulars that we see also differ in degree of both substantiality and being. In addition to ordinary material substances like tomatoes, we see flashes, flames, reflections, and rainbows, on the one hand, and shadows and holes, on the other. Flashes of lightning are unusual events. They are colored, but most events are not colored despited having colored participants. (What color was the Battle of Kosovo?) But what is presently important is Nietzsche's insight that that they also lack substrata. And the plant's shadow is constituted by a relative decrease in the amount of light in a region determined by the positions of the plant, the light source, and the surface on which the shadow is cast. But a relative decrease is a privation or diminution of being. Shadows may be creatures of darkness, but perception makes shadows, a species of privation, the subject matter of demonstrative thought and talk. Despite being particulars, the objects of perception differ, in this way, in mode, substantiality, and being.

Vision makes spatially distant particulars visually present. When I see the yellowish red tomato, I see a particular spatially distant from me. Moreover, I experience the yellowish red as inhering in the external surface of that particular. Of course, not all colors inhere in substances or their parts. A flash can be blue even though the flash lacks a substrata in which the blue could inhere. But even if the blue of the flash inheres in nothing, I experience the blue in the remote spatiotemporal region of the flash. 

\change

Thus, Broad writes:
\begin{quote}
    In its purely phenomenological aspect \emph{seeing} is ostensibly \emph{saltatory}. It seems to leap the spatial gap between the percipient's body and a remote region of space. Then again, it is ostensibly \emph{prehensive} of the surfaces of distant bodies as coloured and extended, and of external events as colour-occurences \emph{localized} in remote regions of space.
\end{quote}
Seeing is saltatory in that the objects of visual awareness are spatially remote; seeing is prehensive in that its objects are present in our visual awareness of them. ``Prehensive'' belongs to a primordial family of broadly tactile metaphors for visual awareness that includes ``grasping'', and ``apprehending''. What unites these metaphors is that they are all a mode of taking in, and ``ingestion'' is a natural variant. The tactile nature of these metaphors can mislead, however, if we take too seriously the contact involved in taking in an object. Thus Broad remarks that ``It is a natural, if paradoxical, way of speaking to say that seeing seems to `bring us into \emph{contact} with \emph{remote} objects' and to reveal their shapes and colors''. The air of paradox, however, is removed once we recognize that the sense in which visual awareness brings us into contact with particulars does not conflict with the sense in which these particulars are remote. To be sure, seeing the tomato a meter away does not make the tomato proximate, but that just means that the sense in which the tomato is present in visual awareness is not the same sense as the tomato being spatially present.

\change

The phenomenology of being visually presented with colors inhering in external bodies played a role in Russell's felt exuberance in abandoning idealism:
\begin{quote}
	I felt it, in fact, as a great liberation, as if I had escaped from a hothouse on to a wind-swept headland. I hated the stuffiness involved in supposing that space and time were only in my mind. I liked the starry heavens even better than the moral law, and could not bear Kant’s view that the one I liked best was only a subjective figment. In the first exuberance of liberation, I became a naïve realist and rejoiced in the thought that grass is really green, in spite of the adverse opinions of all philosophers from Locke onwards.
\end{quote}
What Russell rejoiced in was warrantedly taking external bodies to be the way his perceptual experience presents them to be---as having colors inhering in the opaque surfaces they present. He rejoiced in the greenness of the grass revealed by sight. The phenomenology of being presented with colors inhering in external bodies is what renders this response intelligible.

\change

Visual perception involves a sensory mode of awareness that takes external particulars as objects. This is a claim about sense perception, not sense experience (understood as the genus of which sense perception is a species). It echoes a conception of perception common among early twentieth century realists. There are, of course, important differences between the present conception and its early twentieth century precursors, just as there were important differences between the Cambridge and Oxford realists. Thus, contra the present conception, Russell held that we are acquainted with universals as well as particulars. And Russell and Price held that the distinctive sensory mode of awareness was operative not just in sense perception but in sense experience more generally. In contrast, the Oxford realists restricted sensory awareness to perception. For present purposes, it suffices to accept only what is common to Cambridge and Oxford realism---that perception involves a sensory mode of awareness whose objects exist independently of that awareness. 

Sellars observes that if sensory awareness takes particulars as objects, then it cannot be a form of propositional knowledge. What's known are thoughts, propositions, facts\----\-not par\-ti\-cu\-lars. Sellars does not give us a reason for distinguishing thoughts and particulars in this way. Perhaps he regards the distinction as evident. Prichard, however, gives us a reason, indeed, the right kind of reason. According to Prichard, thoughts have a kind of generality that precludes them from being particulars:
\begin{quote}
	There seems to be no way of distinguishing perception and conception as the apprehension of different realities except as the apprehension of the individual and the universal respectively. Distinguished in this way, the faculty of perception is that in virtue of which we apprehend the individual, and the faculty of conception is that power of reflection in virtue of which a universal is made the explicit object of thought.
\end{quote}
Prichard's claim about the faculty of conception is what provides a reason for distinguishing thoughts and particulars, and Sellars could accept that reason even if he would reject Prichard's claim about the faculty of perception.

But suppose we accept Prichard's claim that ``the faculty of perception is that in virtue of which we apprehend the individual''. If visual awareness takes particulars as objects, it is not a form of propositional knowledge. But that does not mean that vision, so conceived, lacks epistemic significance. Vision can be a source of knowledge insofar as the perceiver can recognize the object of perception for what it is. When I look at the ripening tomato, the tomato is present in my awareness of it. Moreover, if I possess the appropriate recognitional capacities, in being so aware of the ripening tomato, I can come to know various things about it---that it is yellowish red, say. The nonpropositional sensory mode of awareness involved in visual perception may not be a form of knowledge, but it makes me \emph{knowledgeable} of its object. Sensory awareness makes the subject knowledgeable of its object in the sense that it makes knowledge about its object \emph{available} to the perceiving subject. Perception constitutes a change in the subject's knowledge potential whether or not such knowledge is in fact activated.

In looking at the ripening tomato by my window, I see the yellowish red of the tomato. If I recognize what I thus see, I can come to know that the tomato is yellowish red. What I know is a proposition---that the tomato is yellowish red. What I see is a particular---the yellowish red of the tomato. How can awareness of that particular make available knowledge of that proposition? Knowledge is a factive attitude born only to true propositions. The objects of perceptual awareness are epistemically significant because of an alethic connection between them and potentially known propositions. The yellowish red of the tomato is a \emph{truthmaker} of the proposition that the tomato is yellowish red. It is impossible for the yellowish red of the tomato to exist and the proposition that the tomato is yellowish red to be false. Visual awareness is epistemically significant insofar as its object is the truthmaker of potentially known propositions. Vision makes me knowledgeable of particulars arrayed in a mind-independent environment by making me aware of sensible truthmakers.

Sellars famously connects knowledge and reasons in the following fashion:
\begin{quote}
	The essential point is that in characterizing an episode or state as that of \emph{knowing}, we are not giving an empirical description of that episode or state; we are placing it in the logical space of reasons, of justifying and being able to justify what one says.
\end{quote}
Sellars undertakes a number of separable commitments in this passage, but suppose at least this much---that we can only know what we have reason to judge. It is natural to think that vision is a potential source of knowledge only insofar as it makes the subject aware of what reasons there are. But now suppose further that reasons must have a propositional structure. Insofar as visual awareness takes particulars as its objects it lacks a propositional structure. But this would preclude it from the space of reasons. Seeing the tomato, at least as presently conceived, could not be an awareness of what reasons there are and so could not be a potential source of knowledge.

Not all philosophers accept that knowledge requires the possession of reasons. But even granting that knowledge involves reasons, the argument can be resisted insofar as it involves a substantive and controversial claim about the metaphysics of reasons---that reasons must have a propositional structure. Against this, let me dogmatically assert my adherence to a radically externalist conception of reasons. The conception is \emph{externalist} in that, according to it, reasons need not be propositional attitudes or any other psychological state of a subject. External reasons may not be propositional attitudes, but, for all that has been said, they may yet have a propositional structure. Suppose reasons are facts. Facts are not psychological states of a subject. Nevertheless, facts have the structure of the true propositions that represent them. What makes the present conception \emph{radically} externalist is that, according to it, nonpsychological reasons need not have a propositional structure.

On this conception, the yellowish red of the tomato is a reason for thinking that the tomato is not quite ripe. The yellowish red of the tomato lacks a propositional structure---it is a particular, a property instance. It is spatially distant from me---the yellowish red inheres in the opaque surface of the tomato a meter away and inherits its location from the surface in which it inheres. It is an aspect of how things are independently of me. The yellowish red of the tomato is a reason that warrants judging that the tomato is yellowish red. Indeed, in this instance, there could be no better reason---the yellowish red of the tomato warrants judging that the tomato is yellowish red because the former makes true the latter. Of course, not all reasons are of this form. The yellowish red of the tomato is also a reason, in certain circumstances, for thinking that the tomato is unripe, but the yellowish red of the tomato does not make it true that the tomato is unripe; at best, it is sign or symptom of the fruit's relative maturity. Of course, in order for the yellowish red of the tomato to rationally bear on what I am to think, it must be cognitively accessible. But that is what perception does for me---perception makes me aware of what reasons there are. 

In seeing a particular, the subject is in a position to know certain things about it. Being thus knowledgeable endows the subject with a kind of authority. This authority is a \emph{power}---the subject can possess this authority without exercising it. If in seeing the yellowish red of the tomato I am knowledgeable of that particular, then I am warranted in judging that the tomato is yellowish red whether or not I so judge. But if I do, then, at least in propitious circumstances, I thereby come to know that the tomato is yellowish red. The warrant, here, should be understood as an entitlement to judge. It is not a factor in terms of which knowledge could be analyzed or otherwise explained. Moreover, it is an \emph{epistemic} entitlement: The object of my awareness, the yellowish red of the tomato, is a reason that warrants, in the given circumstance, my judging that the tomato is yellowish red where so judging is \emph{coming to know}. Vision confers this epistemic entitlement given the alethic connection between the particular that is the object of visual awareness and the proposition potentially known. Awareness of the sensible truthmaker of a proposition affords the subject with a reason that is in this way akin to proof---it is impossible for the particular to exist and the proposition to be false. Because in seeing the yellowish red of the tomato I possess a reason that would, in the given circumstance, warrant my coming to know that the tomato is yellowish red, I am authoritative about the yellowish red of the tomato. My seeing the yellowish red of the tomato can stand proxy for any inquiry on your part about the color of the tomato. If in coming to know that the tomato is yellowish red, I express my knowledge by stating it, I extend to you an offer to take it on my authority that the tomato is the color that I see it to be. 

Thus ends the dogmatic portion of tonight's presentation.

\change

Perception involves the subject's sensitivity to a mind-independent environment. Moreover, this seems to be a constitutive feature of perception. But it is a further commitment to claim that this sensitivity can be understood independently of the perceiver's sensory awareness. Perception may be a mode of sensitivity, but it does not follow that the perceiver's sensitivity to the environment is a reductively identifiable independent factor in perception, one that can be individuated independently of the perceiver's sensory awareness. Nevertheless, Sellars' conception of sensory impressions commits him to just such a reductive understanding.

There are two aspects of sensory impressions as Sellars conceives of them. First, impressions are conscious states of a subject with a phenomenological character. While impressions are conscious states, there is nothing that they are a consciousness \emph{of}. Impressions are nonintentional. Thus Sellars thinks that it is a conflation:
\begin{quote}
    \ldots\ to assimilate ``having a sensation of a red triangle'' to ``thinking of a celestial city'' and to attribute to the former \ldots\ the `intentionality' of the latter.
\end{quote}
Second, impressions are the effects of the distal environment on the conscious subject. Seeing that something is some way involves a reliable differential responsiveness to objects, events, and properties of the mind-independent environment or facts about these. This reliable differential responsiveness determines no strict correlation. The relationship between distal causes and modifications of the conscious subject is many--one---the same impression involved in seeing that something is the case would be involved in something merely looking to be the case. This reliable differential responsiveness is a species of sensitivity, more generally. In \emph{Science and Metaphysics}, Sellars describes it as ``sheer receptivity''.

Sellars considers two arguments for the claim that perceptual awareness involves impressions. Each corresponds to the two aspects of his conception of them. Sellars argues that the postulation of impressions arises ``in the attempt to explain the facts of sense perception in a scientific style'':
\begin{quote}
    How does it happen that people can have the experience they describe by saying ``It is as though I were seeing a red and triangular physical object'' when either there is no physical object there at all, or, if there is, it is neither red nor triangular? The explanation, roughly, posits that in every case in which a person has an experience of this kind, whether veridical or not, he has what is called a `sensation' or `impression' `of a red triangle'.
\end{quote}

While the explanation ``in a scientific style'' focuses on the phenomenological aspect of sensory impressions, the argument in \emph{Science and Metaphysics} focuses on their environmental sensitivity. McDowell reconstructs Sellars' reasoning as follows:
\begin{quote}
    Sellars thinks that conceptual representations in perception must be guided by manifolds of ``sheer receptivity'', because he thinks that only so can we make it intelligible to ourselves that conceptual occurrences in perceptual experience \ldots\ are constrained by something external to conceptual activity. And as he sees, we need such external constraint in our picture if we are to be entitled to take it that conceptual activity is directed towards an independent reality, as it must be if it is to be intelligible as conceptual activity at all.
\end{quote}

That impression are states of a conscious subject that reliably differentially respond to a mind-independent environment captures the idea that perceptual sensitivity is the \emph{perceiver's} sensitivity to that environment. Perceptual sensitivity is a capacity properly attributable to the perceiving subject. However, impressions are \emph{nonintentional} states of a conscious subject. They take neither environmental particulars nor states of the environment as their object, for they have no object. Since impressions are nonintentional states of a conscious subject with a distinctive phenomenological character, they are individuated independently of the perceiver's sensory awareness of the environment. And since the reliable differential response and its environmental trigger can each be individuated independently of the perceiver's awareness of the environment, and their interaction is a transaction in Nature, Sellars, in conceiving of perceptual sensitivity in terms of manifolds of ``sheer receptivity'', conceives of sensitivity as a reductively identifiable independent factor in perception.  Indeed, the manifolds of ``sheer receptivity'' must enjoy this kind of independence if they are to play their guiding role in the conceptual synthesis of apprehension in intuition.

Thus while Sellars may not be a conjunctivist, he shares an important commitment with the conjunctivist. Conjunctivism is here understood as a claim about sense perception, if not sense experience more generally. According to the conjunctivist, the reliable differential responsiveness to the environment is a necessary if not sufficient condition for perceptual awareness that can be conjoined with other necessary conditions that are not only jointly sufficient but also constitute an \emph{analysis} of perceptual awareness. Then, on pain of circularity, the reliable differential responsiveness must be specified independently of the subject's perceptual awareness. Contrast this with a nonreductive conception of perceptual sensitivity. If perceptual sensitivity is, or is constituted by, perceptual awareness, then perceptual awareness is a mode of sensitivity, but perceptual awareness is not analyzable or otherwise explained in terms of sensitivity reductively understood. Sellars may not be advancing a conjunctive analysis of perception, but he is committed to perceptual sensitivity being a reductively identifiable independent factor in perception given the explanatory role manifolds of ``sheer receptivity'' play in intuition.

Against this, McDowell maintains that while perceptual experience must be guided from without, there is no need for manifolds of ``sheer receptivity'' to play this guiding role:
\begin{quote}
    \ldots\ once we understand how objects can be immediately present to conceptually shaped sensory consciousness in intuition, we can take this need for external constraint to be met by perceived objects themselves.
\end{quote}
McDowell is making negative and positive claims here. The negative claim is a kind of anticonjunctivism---perceptual sensitivity is, or is constituted by, objects in the mind-independent environment being present in sensory awareness in a way that precludes perceptual sensitivity being a reductively identifiable independent factor in perception and, hence, guidance by manifolds of ``sheer receptivity''. The positive claim is that objects in the mind-independent environment are present in sensory awareness in virtue of the sensory actualization of the subject's conceptual capacities. Importantly, these negative and positive claims are logically independent.

Why is it a mistake to conceive of perceptual sensitivity independently of perceptual awareness? Why is anticonjunctivism mandatory?

Consider a counterfactual reduction of perceptual sensitivity. One might try to give such a reduction on the model of Nozick's tracking theory. Roughly speaking, perceptual sensitivity would be the counterfactual covariation of sense experience and the truth of a potentially known proposition \( p \) through a sphere of possibilities that extends to the nearest not-\( p \) world. On the Stalnaker-Lewis semantics, the sphere of possibilities is determined by a conversationally salient similarity metric. The reductive ambitions of the account constrains admissible metrics, however. The relevant similarity metric could not be \emph{perceptual}, if perceptual sensitivity is to be specified independently of perceptual awareness. 

Even if it were immune to counterexample, which of course it isn't, reflection on the best case for the tracking theory undermines its reductive ambitions. In paradigm cases where seeing something can survive small changes, sense experience counterfactually covaries with the truth of a potentially known proposition \emph{because} sense experience affords the perceiver awareness of its environmental subject matter. Suppose my seeing the yellowish red of the tomato is such a paradigm case. My seeing the yellowish red of the tomato would survive small changes in the object, the circumstances of perception, or the perceiver. Thus I would still see the yellowish red of the tomato even if the tomato were slightly smaller, or the illumination were brighter, or I viewed it from a different vantage point. My sense experience counterfactually covaries with the truth of the proposition that the tomato is yellowish red because my sense experience affords me awareness of the yellowish red of the tomato, a particular that makes true that proposition. In the first instance, it is the presence of particulars and not the truth of propositions, that we track in vision. Since perceptual awareness of environmental particulars grounds the counterfactual covariation in paradigm cases, the counterfactual covariation could not be the basis of a reductive understanding of perceptual sensitivity.

\change

While McDowell would be unsurprised by reductive failure, I believe that his anticonjunctivism has deeper roots. I suspect that he is moved by a Hegelian thought that an illicit dualism of mind and world could only be overcome if perceptual sensitivity is, or is constituted by, perceptual awareness. If perceptual sensitivity is understood reductively, it would not unite the subject with the distal object the way perceptual awareness would---in Broad's terminology, it would not be a mode of prehension. It is hard to make this idea precise, but we can begin to get a handle on it by the way it is manifest in McDowell's metaphor of perceived objects ``shaping'' sensory consciousness. Importantly, the metaphor of ``shaping'' should be read in a constitutive rather than a merely causal sense. Whereas St Paul's constitutively shapes the London skyline, by being a contour of that skyline, Nazi bombing shaped it in a merely causal sense. Read constitutively, objects shape the contours of sensory consciousness by \emph{being} the contours of sensory consciousness. Perception would be constitutively linked to the objects in the environment present in sensory awareness. In contrast, sensory impressions are shaped by the environment in a merely causal sense. Indeed this is central to the Platonic metaphor. Just as a stylus impinging on a wax tablet causes an impression, the environment impinging on a subject with the appropriate sensory capacities causes a sensory impression. But perceptual sensitivity is more than just the environment impinging on the state of a conscious subject. The environment shapes sensory consciousness in a constitutive rather than merely causal sense. Perception is guided from without by being constitutively linked to objects in the environment present in sensory awareness. So conceived, perceptual awareness is a mode of sensitivity, but perceptual awareness is not analyzable or otherwise explained in terms of sensitivity reductively understood. Guidance by manifolds of ``sheer receptivity'' is thus precluded.

\change

McDowell's negative claim---that perceptual sensitivity is not a reductively identifiable independent factor in perception---is logically independent of his positive claim---that perception is the sensory actualization of the subject's conceptual capacities. There is nothing in anticonjunctivism that requires our conceptual capacities to be actualized in perceptual awareness. Consider the conception of perception as a mode of taking. Not only is the conception of perception as a nonpropositional sensory mode of awareness \emph{consistent} with anticonjunctivism, but that conception \emph{requires} anticonjunctivism as well. Perception, so conceived, is constitutively linked to the objects of sensory awareness. As such, the guidance is supplied by the objects being present in sensory awareness. Indeed, it provides a reasonable interpretation of the unity of mind and world that perception affords: Perception is a state of the conscious subject that has, as part of its nature, an external particular as a constituent. Perception is a mode of taking in. So conceived, perception is a mode of sensitivity to spatiotemporal particulars arrayed in the mind-independent environment, but the sensory awareness involved in perception is not analyzable or otherwise explained in terms of sensitivity reductively understood.

McDowell's criticism of Sellars is thus susceptible to the following \emph{tu quo que}: 
\begin{quote}
    Perceptual experiences must be guided from without. And indeed they are, I can say. But there is no need for conceptual representations to play this guiding role. In a way that we are now equipped to understand, given the conception of perception as a mode of taking in, that is, as a nonpropositional sensory mode of awareness of external particulars, the guidance is supplied by the objects being present in sensory awareness. Once we understand how objects can be present in sensory awareness, we can take this need for external constraint to be met by perceived objects themselves, irrespective of any conceptual goings-on.
\end{quote}

If perception is a mode of taking in, the objects of perception constitutively shape sensory consciousness. My perception of the tomato is not merely causally or counterfactually linked to the presence of the tomato, my perception is constitutively linked to the perceived tomato. And since my perception is constitutively linked to the tomato, the tomato, itself dappled in sunlight and shadow and partially obscuring the view of the chapel, shapes the contours of my sensory consciousness by being present in that consciousness.

In contrast, McDowell's application of the metaphor is equivocal. Sometimes conceptual representations shape sensory consciousness. Sometimes it is their environmental subject matter. While the latter is nearer the mark, it is hard to understand how the subject matter of conceptual representations, rather than the conceptual representations themselves, can be said to shape sensory consciousness if perception is a sensory actualization of conceptual capacities. It is at the very least unobvious why sensory presentation is a species of conceptual representation. Why must perceptual awareness involve the actualization of conceptual capacities?

\change

That objects in the mind-independent environment are present in sensory consciousness is what drives McDowell's anticonjunctivism. But it is the conceptual character of perception that prevents this from lapsing into the Myth of the Given. Reflection on Kant's thesis:
\begin{quote}
    The unity to the various representations in a judgment also gives unity to a mere synthesis of various representations in an intuition.
\end{quote}
is meant to show how this could be so.

Suppose we take Kant at his word when he claims that ``all combination, be we conscious of it or not, \ldots\ is an act of the understanding''. So conceived, the unity of the sensory manifold is actively provided by the understanding in intuition. Against this  Prichard was right to claim that ``the act of combination cannot confer upon them or introduce into them a unity which they do not already possess''. In seeing the ripening tomato, the object of my sensory awareness, a particular material substance, already enjoys a substantial unity. This particular, with its substantial unity, is a constituent of a state of my consciousness. There is no need for synthesizing activity to provide for any unity in intuition. My sensory awareness already presents me with a substantial unity, the ripening tomato. If perception is a mode of taking in, the objects of sensory awareness are ``selectively made present, but not synthesized'' by our sensibility. From this perspective, any felt need for synthesizing activity to confer unity upon sensory manifolds already represents a withdrawal from the world of sensible particulars. For if a particular were genuinely present, any needed unity would then be provided by the perceived particular itself.

Instead of unity being conferred in intuition by synthesizing activity, McDowell claims that the unity is given in intuition. Moreover, as we have seen, he must maintain this, if his anticonjunctivism is to be sustained. According to McDowell, Kant's thesis does not require that understanding be active in intuition; rather, it requires only that the given be already in a form that could figure in the content of judgment. Compare now McDowell's interpretation of the general framework of givenness:
\begin{quote}
	Givenness in the sense of the Myth would be an availability for cognition to subjects whose getting what is supposedly Given to them does not draw on capacities required for the sort of cognition in question. 
\end{quote}
It is only if the object is given in perception in a form that could figure in the content of discursive activity, paradigmatically, in judgment or assertion, that the Myth of the Given is avoided. For it is only if the object of perception is given in a form that could figure in discursive activity that perception can make available knowledge of that object. 

One cannot be given what one lacks the capacity to receive. On at least that much, we can all agree. However, it is both substantive and controversial to claim that givenness, in the non-Mythical sense, \emph{requires} the unity-providing function of a faculty of discursive activity, even in the minimal sense of the given being already in a form that could figure in the content of discursive activity. After all, I can say in good Sellarsian fashion, there is an alternative---the conception of perception as a mode of taking in. That a coherent, natural alternative exists undermines the thought that givenness, in the non-Mythical sense, \emph{requires} being already in a form that could figure in the content of discursive activity. It is at least unobvious that it does. Moreover, a doubt about how, according to McDowell, perception could be both nondiscursive and conceptual motivates this alternative.

The paradigm of discursive activity is assertion, an overt public performance. Judgment is discursive since it is the inner analogue of assertion. Both judgment and assertion are ways of making something explicit. Since discursive acts are ways of making explicit, the contents of discursive acts are, if you like, articulations. However, according to McDowell, the content given in intuition is unarticulated, if articulable. Since the content of intuition is \emph{unarticulated}, it is not the content of a discursive act. Since the content of intuition is \emph{articulable}, it already has a form that could figure in the content of discursive acts and is, in that sense, conceptual. However, the sense in which the content of perception can reasonably be said to be unarticulated conflicts with its already having a form that could figure in the content of discursive activity.

\change

Consider how McDowell's conception of the unarticulated given converges on the conception C.I.\ Lewis develops in \emph{Mind and the World Order}. Just as McDowell aims to reconcile what is correct in Kant's idealist philosophy with an Aristotelian realism, Lewis aims to reconcile what is correct in Royce's idealist philosophy with the realism of his day. And just as McDowell thinks that vision presents us with visible characteristics that we are not equipped to predicate, the given, as Lewis conceives of it, is incompletely describable since ``in describing it, in whatever fashion, we qualify it by bringing it under some category or other, select from it, emphasize aspects of it, and relate it in particular and avoidable ways.'' It is on these grounds that Lewis claims that the given is ineffable. But being ineffable is precisely to lack a form that could figure in the content of discursive activity.

McDowell can agree that in describing, in whatever fashion, what is given in perception, we qualify it by bringing it under some category or other---that is just the predicative component of perceptual judgment. He can also agree that we select from what is given in perception, at least in the sense of selecting an aspect of its discursive form to articulate. In articulating an aspect of the unarticulated content, we emphasize that aspect, at least over those other aspects that remain unarticulated. Moreover, it often happens that we relate the object of perception in particular and avoidable ways. But McDowell would claim that Lewis goes too far in concluding, on this basis, that the given is ineffable. Every aspect of the unarticulated given is articulable, even if not every aspect is, or could be, articulated.

But the grounds that Lewis offers for the ineffability of the given can be understood in a way that is not so easily dismissed. Lewis can be read as echoing and elaborating Prichard's distinction between thoughts and particulars. The first ground that Lewis offers is Prichard's---having a form that could figure in the content of discursive activity is to have a generality that precludes particulars. In describing, in whatever fashion, what is given in perception, we qualify it by bringing it under some category or other. In bringing a particular that we see under some category we relate it to a range of cases, those cases where the category correctly applies. Lewis also emphasizes the way in which a particular can exceed what is represented in thought, just as what is represented in thought can exceed what is present in the particular. But it is the generality of thought that grounds the categorical distinction, and it is this which is not so easily dismissed. If thoughts are categorically distinct from particulars, and particulars are given in perception, then what is given in perception is not so much unarticulated as it is inarticulable. The yellowish red of the tomato may be the object of my visual awareness, but it is thoughts about the color of the tomato, and not the color of the tomato itself, that are articulable in judgment and assertion. Vision is, if not blind, then dumb.

\change

Consider the striking \emph{aporia} of McDowell's talk of carving out an aspect of the unarticulated if articulable given. Carving is most naturally read as imposing form, but an object with a form that could figure in discursive activity is meant to be selectively presented, not synthesized, by visual sensibility. Moreover, the metaphor of carving is best understood as analyzing a content of a possible judgment in something like Frege's sense. So understood, to carve out an aspect of the unarticulated content is to determine the content of a predication under some appropriate analysis of the unarticulated content. But consider the way a visible characteristic of an object, the yellowish red of the tomato, say, is manifest in sensory consciousness. The yellowish red of the tomato is manifest in sensory consciousness by being present in that consciousness. The exemplification of color properties is among the aspects of the mind-independent environment that vision presents. The yellowish red of the tomato is distinct from the content of a possible predication. But according to McDowell, the visible characteristic is an aspect of the unarticulated content of intuition that is isolable and can be the meaning of a linguistic expression which would be the means for making that content explicit. McDowell misconceives sensory presentation as conceptual representation, in part, by conflating property exemplification with the content of a possible predication. The former is a world-bound property instance, the latter is an intension spread across modal space. The latter may be a means of representing the former, but only the former is present in sensory consciousness.

The \emph{aporia} arises because the only reasonable sense in which the given is unarticulated conflicts with its having a form that could figure in the content of discursive activity. The yellowish red of the tomato that I see is unarticulated only in the vacuous sense of being inarticulable. Seeing the yellowish red of the tomato may make thoughts about that color available to me. But it is thoughts about the color of the tomato, and not the color of the tomato itself, that are articulable in judgment and assertion. Avoiding the Myth of the Given, if it is indeed a myth, does not require the given being already in a form that could figure in the content of discursive activity. Conceptual capacities need not be actualized in perception for the subject in undergoing a perceptual experience to be knowledgeable of the world without the mind.

\change

\end{document}