%!TEX TS-program = xelatex 
%!TEX TS-options = -synctex=1 -output-driver="xdvipdfmx -q -E"
%!TEX encoding = UTF-8 Unicode
%
%  mog
%
%  Created by Mark Eli Kalderon on 2010-04-06.
%  Copyright (c) 2010. All rights reserved.
%

\documentclass[12pt]{article} 

% Definitions
\newcommand\mykeywords{given, Sellars, perception, reasons} 
\newcommand\myauthor{Mark Eli Kalderon} 
\newcommand\mytitle{Before the Law}

\input{preamble/preamble.tex}

%%% BEGIN DOCUMENT
\begin{document}

% Title Page
\maketitle
% \begin{abstract} % optional
% \noindent
% \end{abstract} 
\vskip 2em \hrule height 0.4pt \vskip 2em
% \epigraph{text of epigraph}{\textsc{author of epigraph}} % optional; make sure to uncomment \usepackage{epigraph}

% Layout Settings
\setlength{\parindent}{1em}

% Main Content
\epigraph{Before the law sits a gatekeeper. To this gatekeeper comes a man from the country who asks to gain entry into the law. But the gatekeeper says that he cannot grant him entry at the moment. The man thinks about it and then asks if he will be allowed to come in sometime later on. “It is possible,” says the gatekeeper, “but not now.”---Franz Kafka}

\section{Introduction} % (fold)
\label{sec:introduction}
It is notoriously difficult to understand the distinctive charge of falling prey to the Myth of the Given. The principal obstacle was erected by Sellars himself. The Myth of the Given is said to arise in various forms, and yet each of these forms participates in a framework of givenness---a framework that Sellars declines to specify in general terms. As interpreters of Sellars, we may speculate about how best to understand the general framework of givenness. It is reasonable to attempt to reconstruct that framework in light of Sellars' discussion of the more specific forms of the Myth and especially in light of the conceptual genealogies he offers. However, there is no consensus on how the Myth of the Given is to be understood in general terms even among interpreters philosophically sympathetic with Sellars on fundamental issues. Thus different accounts of the Myth are given by \citet{Brandom:2002fk}, \citet{Coates:2007ly}, \citet{deVries:2005qf}, \citet{McDowell:1996uq,McDowell:1998vn}, \citet{Rosenberg:2007ve}, \citet{Rorty:1979ys}, and \citet{Williams:1977zr}.

One can know or at least judge with relative certainty that a view of perception is a form of the Myth of the Given (if it is indeed a myth) without knowing in general terms what the Myth of the Given is. So, for example, I am increasingly attracted to a view of visual perception where vision affords the perceiver with a sensory mode of awareness of particulars spatially distant from the perceiver's body. This distinctive sensory mode of awareness is non-propositional---it doesn't take a fact as its object, but a particular. Moreover this sensory mode of awareness is fundamental and irreducible. This fundamental and irreducible sensory mode of awareness of spatiotemporal particulars arrayed in a mind-independent environment makes one knowledgeable of that environment. In being so aware of a particular, the perceiver is in a position to know certain things about it, depending, of course, on the perceiver's possession and exercise of recognitional capacities appropriate to the given circumstances. We can be confident that Sellars, at least, would regard this as a form of the Myth of the Given since it involves a pre-conceptual mode of awareness that makes propositional knowledge available to the perceiving subject, whether or not such knowledge is in fact activated (in Williamson's \citeyear{Williamson:1990uq} terminology.) 

Suppose, then, one is liable to the charge of falling prey to the Myth of the Given. How might one respond to the charge? Not directly, it seems. For suppose one could argue that the Myth of the Given was no myth on a specific understanding of that charge. Given the lack of consensus on how the framework of givenness is to be understood, it would remain an open question whether the view defended was yet another form of the Myth on a distinct and potentially superior understanding of it. A lack of a definitive statement of the Myth from Sellars and a lack of consensus on how best to reconstruct the general framework of givenness renders the charge elusive and Protean. Being liable to the charge is like being the man from the country who seeks access to the law in Kafka's parable. Denied access to law, should one simply despair before its gatekeepers?

While no direct response is possible, perhaps an \emph{indirect} response is. I can think of two complementary strategies. First, one might strive to satisfy oneself that perception, so conceived, is epistemically significant---that the view renders intelligible that perception should make one knowledgeable of a mind-independent subject matter. Of course, by itself, this would fail to persuade anyone inclined to prosecute that charge, but persuading the prosecution is unnecessary to achieve the modest task of setting one's own mind at ease about the epistemic significance of perception as one conceives it to be. The task may be modest in this way, but it is an essential first step that promises to clarify what's at issue in the dispute. Second, one might examine rival views of perception that are tailor-made to avoid the Myth of the Given such as those provided by \citet{Sellars:1956xp} and \citet{McDowell:1996uq,McDowell:1998vn,McDowell:2008fk}. If criticism of views designed specifically to avoid the Myth naturally motivates a view that is known or judged with relative certainty to be a form of the Myth, this is some reason, at least, to think that Myth of the Given is no myth. 

These, then, are the strategies that I will be pursuing. In the first part, I proceed dogmatically, sketching how perception, conceived as a non-propositional sensory mode of awareness, can make the subject knowledgeable of particulars without the mind. Doing so will clarify at least one aspect of the issue in dispute---one source of disagreement about the epistemic significance of perception, so conceived, turns on a disagreement about the metaphysics of reasons. In the second part, I proceed dialectically, discussing Sellars and McDowell on perceptual experience. Of particular interest will be McDowell's criticism that Sellars misconceives the nature of sensibility by allowing receptivity to operate independently of the subject's conceptual capacities. There are distinguishable claims made in the course of leveling this criticism. Marking these allows us to see that McDowell is right in rejecting Sellars' account, but wrong in thinking that this requires our conceptual capacities to be operative in perceptual awareness. This involves thinking of the awareness of spatially distant particulars that vision affords the perceiver as pre-conceptual---as a form of the Myth. But if it is, the Myth is no myth. Or so I will argue.

\section{Sensory Awareness} % (fold)
\label{sec:sensory_awareness}

Dissatisfaction with the quality of tomatoes available in central London and not a concern for the veracity of philosophical clichès was my reason for growing tomatoes by my window. It is a large plant in a terra cotta pot that sits on a wooden stool set against a generous Victorian window. In the background is a field of rubble where Middlesex Hospital used to be. Everything has been leveled except for the chapel. Looking up I see a ripening tomato set against this complex scene. It is not quite ripe---it is a yellowish red, if not orange. Moreover, it is dappled in sunlight that has just emerged from the clouds in the aftermath of a shower. When I see the ripening tomato, the ripening tomato is the object of my perceptual experience---I am aware of the tomato in my perceptual experience of it. 

The objects of perception are particulars arrayed in a mind-independent environment. These particulars do not constitute a unified ontological category. Among them are \emph{objects} (the plant, the stool), \emph{events} (a breeze rustling the leaves, the sun's setting), and \emph{property instances} (the yellow-red of the tomato). But objects, events, and property instances have distinctive modes of being. Thus, for example, events are spatiotemporal particulars that unfold through time in a way that neither objects nor property instances do. The particulars also differ in degree of both substantiality and being. In addition to ordinary material substances like tomatoes, we see flashes, flames, reflections, and rainbows, on the one hand, and shadows and holes, on the other hand. Despite differing in mode, substantiality, and being, the objects of perception are all particulars. Moreover, perception makes available demonstrative thoughts about these particulars. Seeing the yellow-red tomato, I recognize it's state of maturation and think---that's not ripe yet. But perception also makes available demonstrative thoughts about flashes---that's red---and holes---that's too big to jump across. 

Vision makes spatially distant particulars visually present. When I see the yellow-red tomato, I see a particular spatially distant from me, and experience the yellow-red as inhering in an external material surface. In the course of describing what is phenomenologically distinctive of seeing, Broad writes:
\begin{quote}
    In its purely phenomenological aspect \emph{seeing} is ostensibly \emph{saltatory}. It seems to leap the spatial gap between the percipient's body and a remote region of space. Then again, it is ostensibly \emph{prehensive} of the surfaces of distant bodies as coloured and extended, and of external events as colour-occurences \emph{localized} in remote regions of space. \citep[32]{Broad:1965dq}
\end{quote}
Broad goes on to remark that ``It is a natural, if paradoxical, way of speaking to say that seeing seems to `bring us into \emph{contact} with \emph{remote} objects' and to reveal their shapes and colors''.

\begin{quote}
	The essential point is that in characterizing an episode or state as that of \emph{knowing}, we are not giving an empirical description of that episode or state; we are placing it in the logical space of reasons, of justifying and being able to justify what one says. \citep[§36]{Sellars:1956xp}
\end{quote}

% section sensory_awareness (end)

\section{McDowell and Sellars on Perceptual Experience} % (fold)
\label{sec:mcdowell_on_sellars_on_perception}

% section mcdowell_on_sellars_on_perception (end)

% section introduction (end)

% Bibligography
\bibliographystyle{plainnat} 
\bibliography{Philosophy} 

\end{document}